%        File: notes.tex
%     Created: Tue Mar 24 01:00 PM 2015 C
% Last Change: Tue Mar 24 01:00 PM 2015 C
%
\documentclass[a4paper, twocolumn]{report}

\usepackage[]{amsthm} 
\usepackage[]{amsmath} 
\usepackage[]{amssymb} 
\usepackage[draft]{todonotes} 
\setlength{\marginparwidth}{2cm}

\newtheorem{defn}{Definition}
\newtheorem{axm}{Axiom}
\newtheorem{prp}{Proposition}



\title{
  \textsc{University of Oslo} \\
  \textsc{Real Analysis - MAT2400} \\
  \textsc{Spring 2015}
}
\author{Ivar Stangeby}
\date{\today}
\begin{document}
\twocolumn[
\begin{@twocolumnfalse}
\maketitle
\begin{abstract}
This document is going to be a way for me as a student of MAT2400 at the
University of Oslo to gather my thoughts around the course Real Analysis.  I've
struggled with my intuition for this subject, and this is a last ditch effort
to build it all up from scratch.

For this task, I've decided to use the text books written by Terence Tao,
namely Analysis I and II. There is absolutely nothing wrong with the text book
offered at my university, it is brilliantly written, but I have been exposed to
the writings of Terence Tao before, and therefore I wish to give his books a
try.

The structure of this document is going to be me writing down the results
encountered throughout the text books along the proofs I find extra intriguing.
I'm going to attempt to prove the theorems myself, and if I find it reasonable
I'm going to write down my own proof. Included will also be my attempted
solutions to selected exercises. 

This document is mainly for my own good and well being, but if anyone can find
any use from them, then that is great.

\end{abstract}
\end{@twocolumnfalse}
]

\chapter{Starting at the beginning: the natural numbers.}
\label{ch:natural_numbers}

In order for us to start exploring the various properties of the real numbers,
which is what real analysis is concerned with, we are going to have to start
from the very beginning.  That is the natural numbers, denoted $\mathbb{N}$.
From these natural numbers, we can construct the integers, $\mathbb{Z}$, the
rationals $\mathbb{Q}$, the real numbers $\mathbb{R}$, and finally; the complex
numbers $\mathbb{C}$. The latter being the main focus of the subject Complex
Analysis.

\todo{Replace all occurances of $++$ with a proper unary plus sign. To remove wonky spacing.}
\section{The Peano axioms}
One of the most standard ways of defining the natural numbers, is in terms of
the \textit{Peano axioms}. One can also define natural numbers through the
notion of cardinality.

\begin{defn}[Informal]
  A \textit{natural number} is any element of the set
  \begin{equation}
    \notag
    \mathbb{N} = \left\{ 0, 1, 2, 3, 4, \cdots \right\},
  \end{equation}
  which is the set of all the numbers created by starting with 0 and then
  counting forward indefinitely. We call $\mathbb{N}$ the \textit{set of all
  natural numbers}.
\end{defn}

In order for us to rigorously define the set of natural numbers, we're going to
use the two fundamental concepts of \textit{the number 0} and the
\textit{increment operation}.  These will be covered in the Peano Axioms.
We will use $n++$ to denote the \textit{successor} of $n$. 

Starting with the first two: 
\begin{axm}
  0 is a natural number.
\end{axm}
\begin{axm}
  If $n$ is a natural number, then $n++$ is also a natural number.
\end{axm}

Now, in order to avoid having to deal with incredibly long strings of $+$'es.
We're going to use an auxilliary definition.
\begin{defn}
  We define 1 to be the number $0++$, 2 to be the number $(0++)++$, etc.
\end{defn}
We can based off of this, propse the following:

\begin{prp}
  3 is a natural number.
\end{prp}
\begin{proof}
  By Axiom 1, 0 is a natural number.
  It then follows by Axiom 2 that both 1, 2, and 3 are natural numbers.
\end{proof}

In order for us to avoid the problem of having the successive numbers wrap
around to previous numbers, we impose a new axiom, namely:

\begin{axm}
  0 is not the successor of any natural number: i.e., we have $n++ \neq 0$ for
  every natural number $n$.
\end{axm}

We can, equipped with this new axiom, show for example the following:

\begin{prp}
  $4$ is not equal to $0$.
\end{prp}
\begin{proof}
  By definition, $4 = 3++$. By the first two axioms, $3$ is a natural number.
  Thus, since $0$ is not the successor of any natural number, $3++ \neq 0$,
  i.e., $4 \neq 0$. 
\end{proof}

Assuming the following axiom allows us to rule out any behaviour where the
successors wrap around, but not to $0$, i.e., $5++ = 1$.

\begin{axm}
  Different natural numbers must have different successors; i.e., if $n, m$ are
  natural numbers and $n \neq m$, then $n++ \neq m++$. Equivalently, if $n++ =
  m++$, then we must have $n = m$.
\end{axm}

We can now prove extensions of the previous proposition where we do not have zeroes
on the right hand side of the equation.

\begin{prp}
  $6$ is not equal to $2$.
\end{prp}
\begin{proof}
  Assume for contradiction that $6 = 2$. By the previous axiom we must have
  $5++ = 1++$.  Applying the same axiom again, we have $5 = 1$ so that $4++ =
  0++$. But, this leads to a contradiction, because by the same axiom, $4 = 0$.
  This contradicts our previously proven proposition.
\end{proof}

Assume now that we are presented with a weird number system
\begin{equation}
  \notag
  \mathbb{N} = \left\{ 0, 0.5, 1, 1.5, 2, 2.5, 3, 3.5, \dots \right\}.
\end{equation}
Even though this set contains real numbers, which we haven't defined or talked
about yet, it satisfies all the previous axioms. But this is not the number
system we're interested in. We want our set of natural numbers to only be
containing all the numbers that can be directly derived from $0$ just using the
successor operation.

We want to introduce some axiom that does not allow other forms of successors to occur.
Therefore we introduce the following:

\todo{Strictly speaking, this axiom is not an axiom but an \textit{axiom schema}.}
\begin{axm}[Principle of mathematical induction]
  Let $P\left( n \right)$ be any property pertaining to a natural number $n$.
  Suppose that $P\left( 0 \right)$ is true, and suppose that whenever $P\left(
  n \right)$ is true, $P(n++)$ is also true. Then $P(n)$ is true for every
  natural number $n$. 
\end{axm}

We're now equipped with the tools required to deal with propositions of the following form:
\begin{prp}
  A certain property $P(n)$ is true for every natural number $n$.  
\end{prp}
\begin{proof}
  Using induction, we show the base case of $P(0)$. Assume, for the sake of
  induction, that $P(n)$ is true. We now want to show that it has to follow
  that $P(n++)$ also must be true. If this is the case, we have shown, using
  mathematical induction that $P(n)$ is true for every natural number $n$.
\end{proof}
\end{document}



