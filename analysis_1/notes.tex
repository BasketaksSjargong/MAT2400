%        File: notes.tex
%     Created: Tue Mar 24 01:00 PM 2015 C
% Last Change: Tue Mar 24 01:00 PM 2015 C
%
\documentclass[a4paper, twocolumn]{report}

\usepackage[]{amsthm} 
\usepackage[]{amsmath} 
\usepackage[]{amssymb} 

\newtheorem{defn}{Definition}

\title{
  \textsc{University of Oslo} \\
  \textsc{Real Analysis - MAT2400} \\
  \textsc{Spring 2015}
}
\author{Ivar Stangeby}
\date{\today}
\begin{document}
\twocolumn[
\begin{@twocolumnfalse}
\maketitle
\begin{abstract}
This document is going to be a way for me as a student of MAT2400 at the
University of Oslo to gather my thoughts around the course Real Analysis.  I've
struggled with my intuition for this subject, and this is a last ditch effort
to build it all up from scratch.

For this task, I've decided to use the text books written by Terence Tao,
namely Analysis I and II. There is absolutely nothing wrong with the text book
offered at my university, it is brilliantly written, but I have been exposed to
the writings of Terence Tao before, and therefore I wish to give his books a
try.

The structure of this document is going to be me writing down the results
encountered throughout the text books along the proofs I find extra intriguing.
I'm going to attempt to prove the theorems myself, and if I find it reasonable
I'm going to write down my own proof. Included will also be my attempted
solutions to selected exercises. 

This document is mainly for my own good and well being, but if anyone can find
any use from them, then that is great.

\end{abstract}
\end{@twocolumnfalse}
]

\chapter{Starting at the beginning: the natural numbers.}
\label{ch:natural_numbers}

In order for us to start exploring the various properties of the real numbers,
which is what real analysis is concerned with, we are going to have to start
from the very beginning.  That is the natural numbers, denoted $\mathbb{N}$.
From these natural numbers, we can construct the integers, $\mathbb{Z}$, the
rationals $\mathbb{Q}$, the real numbers $\mathbb{R}$, and finally; the complex
numbers $\mathbb{C}$. The latter being the main focus of the subject Complex
Analysis.

\section{The Peano axioms}
One of the most standard ways of defining the natural numbers, is in terms of
the \textit{Peano axioms}. One can also define natural numbers through the
notion of cardinality.

\begin{defn}
  A \textit{natural number} is any element of the set
  \begin{equation}
    \notag
    \mathbb{N} = \left\{ 0, 1, 2, 3, 4, \cdots \right\},
  \end{equation}
  which is the set of all the numbers created by starting with 0 and then
  counting forward indefinitely. We call $\mathbb{N}$ the \textit{set of all
  natural numbers}.
\end{defn}

\end{document}



