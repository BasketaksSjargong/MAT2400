%        File: notes.tex
%     Created: Tue Mar 24 01:00 PM 2015 C
% Last Change: Tue Mar 24 01:00 PM 2015 C
%
\documentclass[a4paper, twocolumn]{report}

\usepackage[]{amsthm} 
\usepackage[]{amsmath} 
\usepackage[]{amssymb} 
\usepackage[]{hyperref} 
\usepackage[]{enumerate} 
\usepackage[]{todo} 

\newcounter{dummy} \numberwithin{dummy}{section}
\newcounter{exercise} \numberwithin{exercise}{section}
\newcounter{axmcntr} \numberwithin{axmcntr}{chapter}
\newtheorem{axm}[axmcntr]{Axiom}
\newtheorem{prp}[dummy]{Proposition}
\newtheorem{asm}[axmcntr]{Assumption}
\newtheorem{lma}[dummy]{Lemma}
\newtheorem{crl}[dummy]{Corollary}
\newtheorem*{crl*}{Corollary}
\theoremstyle{definition}
\newtheorem{defn}[dummy]{Definition}
\newtheorem{exc}[exercise]{Exercise}
\makeatletter
\def\@endtheorem{\qed\endtrivlist\@endpefalse }
\makeatother
\newtheoremstyle{solution}{}{}{}{}{\scshape}{:}{ }{}
\theoremstyle{solution}
\newtheorem*{sltn}{Solution}

\newcommand{\dplus}{{+}{+}} % Double unary plus sign.
\newcommand{\union}{\cup}


\title{
  \textsc{University of Oslo} \\
  \textsc{Real Analysis - MAT2400} \\
  \textsc{Spring 2015}\\
}
\author{Ivar Stangeby}
\date{\today}

\begin{document}

\twocolumn[
  \begin{@twocolumnfalse}
    \maketitle
    \begin{abstract}
      This document is going to be a way for me as a student of MAT2400 at the
      University of Oslo to gather my thoughts around the course Real Analysis.  I've
      struggled with my intuition for this subject, and this is a last ditch effort
      to build it all up from scratch.

      For this task, I've decided to use the text books written by Terence Tao,
      namely Analysis I and II. There is absolutely nothing wrong with the text book
      offered at my university, it is brilliantly written, but I have been exposed to
      the writings of Terence Tao before, and therefore I wish to give his books a
      try.

      The first part of this document related to the book Analysis I, is going to be
      mostly involved in building a solid foundation for the concepts discussed in
      Analysis II. The material included from Analysis I is very similar to the
      curriculum of the subject MAT1140 at the University of Oslo.

      The structure of this document is going to be me writing down the results
      encountered throughout the text books along the proofs I find extra intriguing.
      I'm going to attempt to prove the theorems myself, and if I find it reasonable
      I'm going to write down my own proof. Included will also be my attempted
      solutions to selected exercises. 

      This document is mainly for my own good and well being, but if anyone can find
      any use from them, then that is great.
    \end{abstract}
  \end{@twocolumnfalse}
]
\chapter{Introduction}
\chapter{Starting at the beginning: the natural numbers.}
\label{ch:natural_numbers}

In order for us to start exploring the various properties of the real numbers,
which is what real analysis is concerned with, we are going to have to start
from the very beginning.  That is the natural numbers, denoted $\mathbb{N}$.
From these natural numbers, we can construct the integers, $\mathbb{Z}$, the
rationals $\mathbb{Q}$, the real numbers $\mathbb{R}$, and finally; the complex
numbers $\mathbb{C}$. The latter being the main focus of the subject Complex
Analysis.

\section{The Peano axioms}
One of the most standard ways of defining the natural numbers, is in terms of
the \textit{Peano axioms}. One can also define natural numbers through the
notion of cardinality.

\begin{defn}[Informal]
  \label{defn_211}
  A \textit{natural number} is any element of the set
  \begin{equation}
    \notag
    \mathbb{N} = \left\{ 0, 1, 2, 3, 4, \cdots \right\},
  \end{equation}
  which is the set of all the numbers created by starting with 0 and then
  counting forward indefinitely. We call $\mathbb{N}$ the \textit{set of all
  natural numbers}.
\end{defn}

In order for us to rigorously define the set of natural numbers, we're going to
use the two fundamental concepts of \textit{the number 0} and the
\textit{increment operation}.  These will be covered in the Peano Axioms.  We
will use $n\dplus$ to denote the \textit{successor} of $n$.\footnote{When I've
previously encountered the successor of a natural number, it has been described
in terms of a successor function $S$, where $S(n)$ denotes the successor of
$n$.}

Starting with the first two: 
\begin{axm}
  \label{axm_21}
  0 is a natural number.
\end{axm}
\begin{axm}
  \label{axm_22}
  If $n$ is a natural number, then $n\dplus$ is also a natural number.
\end{axm}

Now, in order to avoid having to deal with incredibly long strings of $+$'es.
We're going to use an auxilliary definition.
\addtocounter{dummy}{1}
\begin{defn}
  \label{defn_213}
  We define 1 to be the number $0\dplus$, 2 to be the number $(0\dplus)\dplus$, etc.
\end{defn}
We can based off of this, propse the following:

\begin{prp}
  \label{prp_214}
  3 is a natural number.
\end{prp}
\begin{proof}
  By Axiom 1, 0 is a natural number.
  It then follows by Axiom 2 that both 1, 2, and 3 are natural numbers.
\end{proof}

In order for us to avoid the problem of having the successive numbers wrap
around to previous numbers, we impose a new axiom, namely:

\begin{axm}
  \label{axm_23}
  0 is not the successor of any natural number: i.e., we have $n\dplus \neq 0$ for
  every natural number $n$.
\end{axm}

\addtocounter{dummy}{1}

We can, equipped with this new axiom, show for example the following:

\begin{prp}
  \label{prp_216}
  $4$ is not equal to $0$.
\end{prp}
\begin{proof}
  By definition, $4 = 3\dplus$. By the first two axioms, $3$ is a natural number.
  Thus, since $0$ is not the successor of any natural number, $3\dplus \neq 0$,
  i.e., $4 \neq 0$. 
\end{proof}
\addtocounter{dummy}{1}
Assuming the following axiom allows us to rule out any behaviour where the
successors wrap around, but not to $0$, i.e., $5\dplus = 1$.
\begin{axm}
  \label{axm_24}
  Different natural numbers must have different successors; i.e., if $n, m$ are
  natural numbers and $n \neq m$, then $n\dplus \neq m\dplus$. Equivalently, if $n\dplus =
  m\dplus$, then we must have $n = m$.
\end{axm}

We can now prove extensions of the previous proposition where we do not have zeroes
on the right hand side of the equation.

\begin{prp}
  \label{prp_218}
  $6$ is not equal to $2$.
\end{prp}
\begin{proof}
  Assume for contradiction that $6 = 2$. By the previous axiom we must have
  $5\dplus = 1\dplus$.  Applying the same axiom again, we have $5 = 1$ so that $4\dplus =
  0\dplus$. But, this leads to a contradiction, because by the same axiom, $4 = 0$.
  This contradicts our previously proven proposition.
\end{proof}

\addtocounter{dummy}{1}
Assume now that we are presented with a weird number system
\begin{equation}
  \notag
  \mathbb{N} = \left\{ 0, 0.5, 1, 1.5, 2, 2.5, 3, 3.5, \dots \right\}.
\end{equation}
Even though this set contains real numbers, which we haven't defined or talked
about yet, it satisfies all the previous axioms. But this is not the number
system we're interested in. We want our set of natural numbers to only be
containing all the numbers that can be directly derived from $0$ just using the
successor operation.

We want to introduce some axiom that does not allow other forms of successors to occur.
Therefore we introduce the following:

\begin{axm}[Principle of mathematical induction]
  \label{axm_25}
  Let $P\left( n \right)$ be any property pertaining to a natural number $n$.
  Suppose that $P\left( 0 \right)$ is true, and suppose that whenever $P\left(
  n \right)$ is true, $P(n\dplus)$ is also true. Then $P(n)$ is true for every
  natural number $n$. 
\end{axm}

\addtocounter{dummy}{1}
We're now equipped with the tools required to deal with propositions of the following form:
\begin{prp}
  \label{prp_2111}
  A certain property $P(n)$ is true for every natural number $n$.  
\end{prp}
\begin{proof}
  Using induction, we show the base case of $P(0)$. Assume, for the sake of
  induction, that $P(n)$ is true. We now want to show that it has to follow
  that $P(n\dplus)$ also must be true. If this is the case, we have shown, using
  mathematical induction that $P(n)$ is true for every natural number $n$.
\end{proof}

The previous five axioms are known as the \textit{Peano Axioms} for the natural
numbers.  We now want to more rigorously define the kind of number system we
are to refer to as the \textit{natural numbers}.

\begin{asm}[Informal]
  \label{asm_26}
  There exists a number system $\mathbb{N}$, whose elements we will call
  \textrm{natural numbers}, for which Axioms 1-5, are true.
\end{asm}

\addtocounter{dummy}{1}
\addtocounter{dummy}{1}
\addtocounter{dummy}{1}
\addtocounter{dummy}{1}

This number system is what we refer to as \textit{the} natural number system.
But one should not rule out the possibility that there are more than one
natural number system. But as long as these are \textit{isomorphic} one can
consider them as equal.

With only this, rather simplistic definition of natural numbers, the five
axioms, and some axioms from set theory we can build all other number systems,
create functions and do the algebra and calculus that we are used to.

A very common question now arises, and this is about the finiteness or
infiniteness of the natural number system.  How can something infinite come
from something strictly finite?  One can easily show that all the natural
numbers are finite. It is clear that $0$ is finite. If $n$ is finite, then
clearly $n\dplus$ is finite. Therefore all natural numbers are finite.  It then
follows that infinity is not a natural number. There are other number systems
that admit the infinite numbers.  

It is an interesting fact that the definition of $\mathbb{N}$ is
\textit{axiomatic} rather than \textit{constructive}. This means, that so far
we're only conserned with what the natural numbers are, not what they do, what
they measure or what they can be used for.

As long as a mathematical model obeys the previous axioms, it is of no concern
whether which mathematical model is ``true''. It is this form of
\textit{abstractness} that makes mathematics so useful. One does not
neccesarily need a concrete model, because the numbers can be understood
abstractly through the use of axioms.

As a consequence of the axioms previously discussed, we can now define
sequences \textit{recursively}.  That is, start with some base value and then
building the next value in the sequence by means of a function.  This leads to
the following:

\begin{prp}[Recursive definitions]
  \label{prp_2116}
  Suppose for each natural number $n$, we have some function $f_n:\mathbb{N}
  \rightarrow \mathbb{N}$ from the natural numbers to the natural numbers. Let
  $c$ be a natural number.  Then we can assign a unique natural number $a_n$ to
  each natural number $n$, such that $a_0 = c$ and $a_{n\dplus} = f_{n}\left( a_n
  \right)$ for each natural number $n$. 
\end{prp}
\begin{proof}
  Using induction, we verify the base case. We clearly see that this procedure
  gives a single value to $a_0$, namely $c$. (We know from axiom 3 that $a_0$
  won't be redefined.) Suppose now inductively that the procedure gives a
  single value to $a_n$. Then it gives a single value to $a_{n\dplus}$, namely
  $a_{n\dplus} = f_n\left( a_n \right)$. (We know from axiom 4 that $a_{n\dplus}$ won't
  be redefined.) This completes the induction, since $a_n$ is defined for every
  natural number $n$, with a single value assigned to each $a_n$.
\end{proof}

Equipped with the tools that are recursive definitions we can now define
multiple operations on the set of natural numbers. Up until now, we've only
dealt with one, being the increment operation.

\section{Addition}

Currently our number system does not support any more advanced operations than
incrementing a number.  We now turn our heads to addition. The operation is
simple. To add $3$ to $5$, we simply increment $5$ three times. This is one
increment more than adding $2$ to $5$, which is one increment more than adding
$1$ to $5$, which is one increment more than adding $0$ to $5$. We can
therefore easily give a recursive definition of addition.

\begin{defn}[Addition of natural numbers]
  \label{defn_221}
  Let $m$ be a natural number. To add zero to $m$, we define $0 + m = m$. Now
  suppose inductively that we have defined how to add $n$ to $m$. Then we can
  add $n\dplus$ to $m$ by defining $(n\dplus) + m = (n + m)\dplus$.        
\end{defn}

For example, $2 + 3 = (3\dplus)\dplus = 4\dplus = 5$. By using the priciple of mathematical
induction, we see that we now have defined $n + m$ for every natural number
$n$. We are specializing the previous general discussion about recursive
definitions to the setting where $a_{n} = n + m$, and $f_{n}\left( a_n \right)
= a_{n}\dplus$.

It's worth noting that this definition is actually \textit{asymmetric}. That
is, while yielding the same result, $3 + 5$ is incrementing $5$ three times,
where as $5 + 3$ is incrementing $3$ five times. We shall soon see, that it is
a general fact that $a + b = b + a$ for all natural numbers $a, b$. 

One can easily prove, using the first two axioms and the principle of
mathematical induction to show that the sum of two natural numbers is again a
natural number.

At the present moment, we have only two facts about addition. However, this is
perfectly sufficient to deduce everything else we know about addition. Starting
with some basic lemmas.

\begin{lma}
  \label{lma_222}
  For any natural number $n$, $n + 0 = n$. 
\end{lma}

This lemma is not obvious from our previous definition, since we still do not
know that $a + b = b + a$. 

\begin{proof}
  Using induction. The base case $0 + 0 = 0$ follows from the definition of
  addition of natural numbers.  $0 + m = m$ for all natural numbers, and $0$ is
  known to be a natural number.  Suppose inductively that $n + 0 = n$. We now
  wish to show that $(n\dplus) + 0 = n\dplus$. By definition of addition yields that
  $(n\dplus) + 0$ is equal to $(n + 0)\dplus$, which is equal to $n\dplus$ since $n + 0 =
  n$. This closes the induction.
\end{proof}

\begin{lma}
  \label{lma_223}
  For any natural numbers $n$ and $m$, $n + (m\dplus) = (n + m)\dplus$.
\end{lma}

Again, this is not obvious. 

\begin{proof}
  We induct on $n$, keeping $m$ fixed. Considering the base case $n = 0$. We
  therefore have to prove $ 0 + (m\dplus) = (0 + m)\dplus$. By definition of
  addition, we have $0 + (m\dplus) = m\dplus$ and $0 + m = m$. So, both sides
  are equal to $m\dplus$ and are thus equal. Assuming now, that we have shown
  $n + (m\dplus) = (n + m)\dplus$, we want to show that $(n\dplus) + (m\dplus)
  = \left( \left( n\dplus \right) + m \right)\dplus$.  Looking at the left hand
  side. By definition of addition it is equal to $\left( n + \left( m\dplus
  \right) \right)\dplus$, which in turn, by the inductive hypothesis, is equal
  to $\left(\left( n + m \right)\dplus\right)\dplus$. Now, examining the right
  hand side. By the definition of addition, it is equal to $\left(\left( n + m
  \right)\dplus\right)\dplus$. The two sides are equal, and this closes the
  induction.
\end{proof}
We can now easily show the following result:
\begin{crl*}
  \label{crl_plusone}
  For all natural numbers $n$, $n\dplus = n + 1$.    
\end{crl*}
\begin{proof}
  This is a special case of the previous lemma, where $m = 0$.  We have $n +
  (m\dplus) = (n + m)\dplus$. Setting $m = 0$, we get $n + (0\dplus) = (n +
  0)\dplus$. Evaluating the left hand side we get $n + 1$ and evaluating the
  right hand side, we get $(n)\dplus = n\dplus$, which is what we wanted to
  show.
\end{proof}

Now for one of the first major results. We can now prove that $a + b = b + a$.

\begin{prp}[Addition is commutative]
  \label{prp_224}
  For any natural numbers $n$ and $m$, $n + m = m + n$. 
\end{prp}
\begin{proof}
  Using induction on $n$ keeping $m$ fixed.  First showing the base case, where
  $n = 0$.  We want to show that $0 + m = m + 0$.  By the definition of
  addition, the left hand side is equal to $m$. By lemma 2.2.2, the right hand
  side is equal to $m$. Therefore, the base case is true.  Now, assuming that
  it is shown that $n + m = m + n$.  We now want to show that $(n\dplus) + m =
  m + (n\dplus)$.  Looking at the left hand side, we see that it is equal to
  $(n + m)\dplus$, by definition of addition.  The right hand side, by lemma
  2.2.3 must be equal to $(m + n)\dplus$. Since we assumed $n + m = m + n$, the
  left and right hand side is equal and therefore the induction is closed.
\end{proof}

\begin{prp}[Addition is associative]
  \label{prp_225}
  For any natural numbers $a, b, c$, we have $(a + b) + c = a + (b + c)$.  
\end{prp}
\begin{proof}
  See Exercise 2.2.1.
\end{proof}

\begin{prp}[Cancellation law]
  \label{prp_226}
  Let $a, b, c$ be natural numbers such that $a + b = a + c$. 
\end{prp}

Since we haven't explored the concept of subtraction or negative numbers yet we
cannot use these properties to prove this law. This law is actually crucial in
defining subtraction rigorously. 

\begin{proof}
  We prove this with induction on $a$, keeping $b$ and $c$ fixed.  Showing the
  base case with $a = 0$. We have $0 + b = 0 + c$. Using the definition of
  addition, $b = c$. Now, for the inductive hypothesis.  Assuming that it is
  shown that $ a + b  = a + c$, we want to show that $(a\dplus) + b = (a\dplus)
  + c$ implies $b = c$.  Left hand side evaluates to $(a+b)\dplus$ and the
  right hand side evaluates to $(a + c)\dplus$ by the definition of addition.
  By Axiom 2.4 we see that $(a + b) = (a + c)$, and therefore by our assumption
  $b = c$. This closes the induction.
\end{proof}

We now want to look at how natural numbers interacts with positivity. First, a definition:
\begin{defn}[Positive natural numbers]
  \label{defn_227}
  A natural number $n$ is said to be \textit{positive} if and only if it is not equal to $0$.
\end{defn}

This leads to the following proposition.
\begin{prp}
  \label{prp_228}
  If $a$ is positive and $b$ is a natural number, then $a + b$ is positive (and hence $b + a$ is also, by Proposition 2.2.4).
\end{prp}
\begin{proof}
  Using induction on $b$. Showing the base case where $b = 0$.  Assuming $a$ a
  positive number and $b$ a natural number.  We then have $ a + b = a + 0 = a$,
  and by assumption $a$ is a positive number.  Now for the inductive step. We
  assume shown that $a + b$ is a positive number.  We want to show that $a + (
  b\dplus)$ must also be a positive number. Using the commutativity of natural
  numbers and the definition of addition we can show that this must be equal to
  $(a + b)\dplus$. Since we know that $0$ is not the successor to any number,
  and that $(a + b)$ is positive, we must have $(a + b)\dplus \neq 0$. This
  closes the induction.
\end{proof}

\begin{crl}
  \label{crl_229}
  If $a$ and $b$ are natural numbers such that $a + b = 0$, then $a = 0$ and $b = 0$. 
\end{crl}
\begin{proof}
  Assume for contradiction that $a \neq 0$ and $b \neq 0$.  Since $a \neq 0$
  then it is positive by definition, and then it follows by \ref{prp_228} that
  $a + b$ is positive.  The same argument for $b \neq 0$. Therefore, our
  assumptionds leads to contradictions. In other words, $a = 0, b = 0$.
\end{proof}

\begin{lma}
  \label{lma_2210}
  Let $a$ be a positive number. Then there exists exactly one natural number
  $b$ such that $b\dplus = a$.
\end{lma}
\begin{proof}
  See Exercise 2.2.2.
\end{proof}

We can now, since we have a notion of addition, proceed with defining a notion
of order on the natural numbers.
\begin{defn}[Ordering of the natural numbers]
  \label{defn_2211}
  Let $n$ and $m$ be natural numbers. We say that $n$ is \textit{greater than
  or equal to} $m$, and write $n \geq m$ or $m \leq n$, if and only if we have
  $n = m + a$ for some natural number $a$.  We say that $n$ is \textit{strictly
  greater than} $m$ and write $n > m$ or $m < n$, if and only if $n \geq m$ or
  $m \leq n$ and $n \neq m$.
\end{defn}

An example would be $8 > 5$ because $ 8 = 5 + 3$ and $ 8 \neq 5$.  Another
important thing to note is that $n\dplus > n$ for all natural numbers $n$.
This means that there are no largest natural number $n$, because the next
number $n\dplus$ is always larger.

\begin{prp}[Basic properties of order for natural numbers]
  \label{prp_2212}
  Let $a, b, c$ be natural numbers. Then
  \begin{enumerate}[(a)]
    \item 
      (Order is reflexive) $a\geq a.$
    \item 
      (Order is transitive) If $a \geq b$ and $b \geq c$, then $a \geq c$.
    \item 
      (Order is anti-symmetric) If $a \geq b$ and $b \geq a$, then $a = b$.
    \item
      (Addition preserves order) $a \geq b$ if and only if $a + c \geq b + c$.
    \item
      $ a < b $ if and only if $ a\dplus \leq b$.
    \item 
      $a < b$ if and only if $ b = a + d$ for some \textrm{positive} number $d$.
  \end{enumerate}
\end{prp}
\begin{proof}
  See Exercise 2.2.3.
\end{proof}

\begin{prp}[Trichotomy of order for natural numbers]
  \label{prp_2213}
  Let $a$ and $b$ be natural numbers. Then exactly one of the following
  statements is true: $a < b, a = b$ or $a > b$.
\end{prp}
\begin{proof}
  The gaps of this proof will be filled in Exercise 2.2.4.

  The first step is going to be showing that no more than one of the statements
  can hold at any given time.  That is, assuming $a < b$, then $a \neq b$ by
  definition. If $a > b$, then $ a \neq b$ by definition.  Assuming $a < b$ and
  $b > a$, we have $ a = b + m$ and $b = a + n$. Substituting we get $ a = a +
  n + m$, and using the cancellation law for addition we get $0 = n + m$. By
  \ref{crl_229} we get $n = 0$ and $m = 0$, thus $a = b$ which is a
  contradiction. Therefore only one of the statements may apply at any given
  time.

  We must now show that at least one of the statemements must apply. Keeping
  $b$ fixed we induct on $a$. Considering the base case where $a = 0$ yields $0
  \leq b$ for all $b$.  In other words, either $0 = b$ or $0 < b$.  For the
  inductive hypothesis we assume shown that at least one of the three holds for
  two natural numbers $a$ and $b$. We now want to show that it must neccesarily
  hold at least one for $a\dplus$ and $b$.

  If $a < b$ then $a\dplus \leq b$, thus $a\dplus = b$ or $a\dplus < b$ by
  \ref{prp_2212} If $a = b$ then $a\dplus = a + 1 = b + 1$ and thus by
  definition of the ordering of natural numbers, we have $a\dplus > b$.  If $a
  > b$ then $a = b + n$ for some number $n$. It then follows that $a\dplus = (b
  + n)\dplus$ which in turn is equal to $b + (n\dplus) = b + (n + 1)$. In other
  words, by definition $a\dplus > b$.  This closes the induction.
\end{proof}

Armed with the previous proposition, we can now obtain a stronger version of
the principle of induction.

\begin{prp}[Strong principle of induction]
  \label{prp_2214}
  Let $m_0$ be a natural number, and let $P(m)$ be a property pertaining to an
  arbitrary natural number $m$. Suppose that for each $m \geq m_0$, we have the
  following implication: if $P(m')$ is true for all natural numbers $m_0 \leq
  m' < m$, then $P(m)$ is also true. (In particular, this means that $P(m_0)$
  is true, since in this case the hypothesis is vacuous.) Then we can conclude
  that $P(m)$ is true for all natural numbers $m \geq m_0$.
\end{prp}
\begin{proof}
  See Exercise 2.2.5.
\end{proof}

\begin{exc}
  Prove \ref{prp_225}. (Hint: fix two of the variables and induct on the third).
\end{exc}

\begin{sltn}
  Fixing $b$ and $c$ we induct on $a$.  Examining the base case where $a = 0$
  yields $(0 + b) + c = 0 + (b + c)$. Evaluating both sides using the definition
  of addition of natural numbers yields $b + c$ on both sides. Therefore the base
  case holds.

  For the inductive hypothesis we assume that we have shown that $( a + b ) + c =
  a + (b + c)$ holds. We want to show that it also holds for $a\dplus$.  We get
  $\left( \left( a\dplus \right) + b \right) + c = \left( a\dplus \right) + (b +
  c)$.  Evaluating the left hand side by applying the definition of addition
  twice, we get
  \begin{align*}
    \left( \left( a\dplus \right) + b \right) + c &= \left( \left(a + b\right)\dplus\right) + c \\
    &= \left( \left( a + b \right) + c \right)\dplus.
  \end{align*}

  Similarily, evaluating the right hand side yields
  \begin{align*}
    \left( a\dplus \right) + (b + c) = (a + (b + c))\dplus.
  \end{align*}

  By our assumption we know $a + (b + c) = (a + b) + c$, and therefore this
  closes the induction.
\end{sltn}

\begin{exc}
  Prove \ref{lma_2210}. (Hint: use induction.)
\end{exc}
\begin{sltn}
  Let $a$ be a positive number. We want to show that there exists exactly one
  natural number $b$ such that $b\dplus = a$.  Fixing $b$ we induct on $a$. We
  first want to show the base case where $a = 1$.  It then follows that if we
  set $b = 0$, then we have $b\dplus = 0\dplus = 1 = a$.  

  For the inductive hypothesis we assume that we have shown that there exists
  exactly one natural number $b$ such that $b\dplus = a$. We want to show that
  it then follows that there exists exactly one natural number $c$ such that
  $c\dplus = a\dplus$.  Substituting this for $a$ we get $c \dplus =
  (b\dplus)\dplus$. From axiom 4 we know that $c = b\dplus$. This closes the
  induction.

\end{sltn}

\begin{exc}
  Prove \ref{prp_2212}. (Hint: You will need many of the preceding propositions, corollaries and lemmas.)
\end{exc}
\begin{sltn}
  Proving basic properties of order for natural numbers.
  \begin{enumerate}[(a)]
    \item $a \geq a$. \\
      [0.2cm]
      From the assumption, we have $a = a$ or $a > a$. For $a > a$ there must
      exist a positive number $ b $ such that $a = a + b$. Using the law of
      cancellation, we get $b = 0$, which contradicts the fact that $b$ is
      positive. Therefore $a = a$, which always holds. In any case, we're done.

    \item If $a \geq b$ and $b \geq c$, then $a \geq c$.\\
      [0.2cm]
      From the assumption, there exists natural numbers $m$ and $n$ such that
      $a = b + m$ and $b = c + n$. Substituting we get $a = (c + n) + m$.
      Using the associativity of addition of we get $a = c + (n + m)$. Since
      $(n + m)$ is a natural number, by definition $ a \geq c$.

    \item If $a \geq b$ and $b \geq a$ then $a = b$. \\
      [0.2cm]
      If $a \geq b$, then there exist a natural number $c$ such that $a = b +
      c$.  Also, if $b \geq a$ then there exists a natural number $d$ such that
      $b = a + d$.  Substituting we get $a = (a + d) + c$. Using the
      associativity of the natural numbers, we have $a = a + (d + c)$. The law
      of cancellation gives $0 = (d + c)$. From \ref{crl_229} we have $d = 0$,
      $c = 0$.  Again, substituting yields $a = 0 + b = b$ and $b = 0 + a = a$.

    \item $a \geq b$ if and only if $a + c \geq b + c$.\\
      [0.2cm]
      We argue contrapositively.  Assume that $a + c < b + c$. By definition,
      then there exists a natural number $d$ such that $b + c = (a + c) + d$.
      Using associativity and the law of cancellation, this simplifies to $b =
      a + d$, which in turn means that $a < b$. We have now shown the
      contrapositive of the right implication. Thus the right implication holds.

      Now to show the left implication.  Assume that $a + c \geq b + c$. We see
      that there exist a natural number $d$ such that $a + c = (b + c) + d$
      (equality when $d = 0$). Using associativity and the law of
      cancellation, we get $a = b + d$.  This is the definition of
      \textit{greater than}, therefore we can conclude that $a \geq b$. 

      We have now shown both implications and the statement holds.

    \item $a < b$ if and only if $a\dplus \leq b$.\\
      [0.2cm]
      Assume $a < b$, then by definition there exists a positive number $c$
      such that $ b = a + c$.  Substituting this into the right hand side we
      achieve $a\dplus \leq a + c$. Rephrased, $a + 1 \leq a + c$. This clearly
      holds true for any positive number $c$. (Can be shown rigorously by
      mathematical induction).

      Now, assume that $a\dplus \leq b$. In other words, there exists a natural
      number $d$ such that $ b = (a\dplus) + d$. ($d$ is allowed to be zero).
      Using the law of associativity we can deduce that $b = a + (1 + d)$. In
      other words, the definition of $ a < b$.

      The two implications have been shown and the statement holds.

    \item $a < b$ if and only if $b = a + d$ for some positive number $d$.\\
      [0.2cm]
      Assume $a < b$. From the definition, we get $a \leq b$ and $ a \neq b$.
      $a \leq b$ means that there must exist a natural number $n$ such that $a
      + n = b$.  In the case where $n = 0$, $a = b$ which is a contradiction of
      our assumption, therefore $n$ must be positive by definition of positive
      since it is not equal to $0$.  Let $d = n$, and we have shown that $a <
      b$ if $b = a + d$ for some positive number $d$.

      Now, assume that $b = a + d$ for some positive number $d$. Since $d$ is
      not $0$ by definition, we know that $b \neq a$. Therefore, we have both $
      a \leq b$ as well as $a \neq b$.  This is the definition of $a < b$, and
      we are done.
  \end{enumerate}
\end{sltn}

\begin{exc}
  Justify the three statements marked (why?) in the proof of \ref{prp_2213} 
\end{exc}

\begin{sltn}
  Filling in the gaps in the proof of \ref{prp_2213}.
  \begin{enumerate}[(a)]
    \item When $a = 0$ we have $0 \leq b$ for all $b$\\
      [0.2cm]
      $a = 0$ means that in the definition of $\leq$ there exist some natural
      number $c$ such that we have $b = 0 + c$.  Using \ref{lma_222} gives $b =
      c$. Thus, just chosing $c$ to be equal to $b$ satisfies the definition of
      $a \leq b$ for all $b$ in the cases where $a = 0$. Therefore the base
      case of the mathematical induction holds.

    \item If $a > b$, then $a\dplus > b$\\
      [0.2cm]
      Assuming $a > b$. From the definition we know we have $a \geq b$ and $a
      \neq b$.  This again means there exists some natural number, $c$,
      different from zero that satisfy $a = b + c$. In other words, a positive
      number $c$.

      Incrementing both sides gives us $a\dplus = (b + c)\dplus$. This gives us
      $a\dplus = b + (c + 1)$, by using the definition of addition and
      associativity. This is the definition of $a\dplus \geq b$, but since $c$
      cannot be $0$, there is no way of obtaining the equality $a\dplus =
      b\dplus$, thus by \ref{axm_24} we have $a \neq b$ which gives us the
      condition we need for setting $a \dplus > b$ which is what we wanted to
      show.

    \item If $a = b$, then $a\dplus > b$.\\
      [0.2cm]
      Assuming $a = b$. This means, that in the definition of $a \geq b$, we
      know that the natural number $c$ that satisfies $a = b + c$ must be zero
      (this follows from the law of cancellation and \ref{lma_222}).
      Incrementing both sides yield $a\dplus = (b+0)\dplus$, which in turn
      gives us $a\dplus = b\dplus = b + 1$.  This satisfies the definition of
      $a\dplus > b$ because there exist a positive number $1$ such that
      $a\dplus = b + 1$.
  \end{enumerate}
\end{sltn}

\begin{exc}
  \todo{Finish this exercise}
  Prove \ref{prp_2214}. (Hint: define $Q(n)$ to be the property that $P(m)$ is
    true for all $m_0 \leq m < n$; note that $Q(n)$ is vacuously true when $n <
  m_0$.)
\end{exc}

\begin{exc}
  Let $n$ be a natural number, and let $P(m)$ be a property pertaining to the
  natural numbers such that whenever $P(m\dplus)$ is true, then $P(m)$ is true.
  Suppose that $P(n)$ is also true.  Prove that $P(m)$ is true for all natural
  numbers $m \leq n$; this is known as the \textit{principle of backwards
  induction}.  (Hint: apply induction to the variable $n$.)
\end{exc}

\begin{sltn}
  Inducting on $n$ and examining the base case where $n = 1$ yields $P(1) =
  P(0\dplus)$ being true.  By the properties of $P(m\dplus)$ it follows that $P(0)$
  is true.

  Now, assume that we have shown this to be true up to $n = d\dplus$, we now want
  to show that it holds for $(d\dplus)\dplus$.  $P((d\dplus)\dplus)$ true tells
  us that $P(d\dplus)$ must be true. Since $P(d\dplus)$ is true $P(d)$ is true,
  and so on and so forth.

  Thus by \textit{backwards induction} we have shown $P(m)$ is true for all
  natural numbers $m \leq n$. 
\end{sltn}

\section{Multiplication}

We have now shown all the basic facts known to be true for the addition and
ordering of natural numbers.  We now introduce multiplication. Just as addition
is iterated incrementation, we can define multiplication to be iterated
addition.

\begin{defn}[Multiplication of natural numbers]
  \label{defn_231}
  Let $m$ be a natural number number. To multiply zero to $m$, we define $0
  \times m = 0$. Now suppose inductively that we have defined how to multiply
  $n$ to $m$. Then we can multiply $n\dplus$ to $m$ by defining $\left( n\dplus
  \right) \times m = (n \times m) + m$.
\end{defn}

An example would be $2 \times m = 0 + m + m$. By induction, it is easily
verified that the product of two natural numbers is also a natural number.

\begin{lma}[Multiplication is commutative]
  \label{lma_232} 
  Let $n, m$ be natural numbers. Then $n \times m = m \times n$. 
\end{lma}
\begin{proof}
  See Exercise 2.3.1.
\end{proof}

We now, in order to ease writing, start abbreviating $n \times m$ as $nm$.
Using the usual rules of precedence there is no ambiguitiy. Therefore $ab + c$
is equal to $(a\times b) + c$. We will also use the usual precedence rules for
the other arithmetic operations as they are defined later.

\begin{lma}[Natural numbers have no zero divisors]
  \label{lma_233}
  Let $n, m$ be natural numbers. Then $n \times m = 0$ if and only if at least
  one of $n, m$ is equal to zero.  In particular, if $n$ and $m$ are both
  positive, then $nm$ is also positive. 
\end{lma}
\begin{proof}
  See Exercise 2.3.2.
\end{proof}

\begin{prp}[Distributive law]
  \label{prp_234}
  For any natural numbers $a, b, c$, we have $a(b+c) = ab + ac$ and $(b+c)a =
  ba + ca$.
\end{prp}

\begin{proof}
  Assume $a, b, c$ to be natural numbers.  Using the commutativity of
  multiplication, we see that $a(b+c) = (b+c)a$.  Therefore we only have to
  show the first case.

  Inducting on $a$ keeping $b$ and $c$ fixed.  The base case, with $a = 0$
  gives us for the left hand side: $0(b+c)$ which is equal to $0$ by definition
  of multiplication. Right hand side equates to $0a + 0b$, which is equal to
  $0$. Therefore the two sides are equal and the statement holds.

  Now, assume for the sake of induction that we have shown $a(b+c) = ab + ac$.
  We now want to show that it must also hold for $(a\dplus)(b+c) = (a\dplus)b +
  (a\dplus)c$.  By definition of mutlitplication, we have that the left side
  equates to $a(b+c) + (b+c)$. Using our assumption this evaluates to $ab + ac
  + (b+c)$.  Using the laws of commutativity we can rearrange this to $(ab + b)
  + (ac + c)$.  From the definition of multiplication this is equal to
  $(a\dplus)b + (a\dplus)c$ which is what we wanted to show. The induction is
  closed.
\end{proof}

\begin{prp}[Multiplication is associative]
  \label{prp_235} 
  For any natural numbers $a, b, c$ we have $(a\times b) \times c = a \times
  (b\times c)$
\end{prp}
\begin{proof}
  See Exercise 2.3.3.
\end{proof}

\begin{prp}[Multiplication preserves order]
  \label{prp_236}
  If $a, b$ are natural numbers such that $a < b$, and $c$ is positive, then
  $ac < bc$. 
\end{prp}
\begin{proof}
  Assuming $a < b$, then by definition $a \leq b$ and $a \neq b$.  Thus, again
  by definition there must exist some natural number $d$ such that $a + d = b$,
  but since $a \neq b$, we must have $d \neq 0$. Therefore $d$ is a positive
  number.  Multiplying by a positive number $c$ we get $(a + d)c = bc$, and
  using the law of commutativity we have $ac + dc = bc$, thus by definition $ac
  \leq bc$. But since $d \neq 0$ we cant have equality, and therefore $ac <
  bc$. 
\end{proof}

\begin{crl}[Cancellation law]
  \label{crl_237}
  Let $a, b, c$ be natural numbers such that $ac = bc$ and $c$ is non-zero.
  Then $a = b$. 
\end{crl}
\begin{proof}
  Initially assume that $ac = bc$.  By the trichotomy of order (\ref{prp_2213}), exactly one of
  the following must hold: $a = b$, $a < b$ or $a > b$.  Assume first that $a <
  b$. By \ref{prp_236} we have $ac < bc$ which is a contradiction.  Similarly,
  assuming $b < a$ we have by the same proposition that $bc < ac$ which also is
  a contradiction.  Therefore, by the trichotomy of order, $a = b$ is the only
  possibility. 
\end{proof}

We can now, since we have all the familiar operations of addition and
multiplication discard the increment operation and just use the fact that
$n\dplus = n+1$ in the cases where we need to describe incrementation.  We're
getting closer and closer to things being the way they are used to.

We're now proposing one of the fundamental theorems in number theory, namely the Euclidean algorithm

\addtocounter{dummy}{1}
\begin{prp}[Euclidean algorithm]
  \label{prp_239}
  Let $n$ be a natural number, and let $q$ be a positive number. Then there
  exist natural numbers $m, r$ such that $0 \leq r < q$ and $n = mq + r$. 
\end{prp}
\begin{proof}
  See Exercise 2.3.5
\end{proof}

\addtocounter{dummy}{1}

Just like addition is iterated incrementation and multiplication is iterated
addition, we can recursively define \textit{exponentiation} as iterated
multiplication.

\begin{defn}[Exponentiation for natural numbers]
  Let $m$ be a natural number. To raise $m$ to the power $0$, we define $m^{0}
  = 1$. Now suppose recursively that $m^{n}$ has been defined for some natural
  number $n$, then we define $m^{n\dplus} = m^{n} \times m$.
\end{defn}

\begin{exc}
  Prove Lemma \ref{lma_232}. (Hint: Modify the proofs of \ref{lma_222},
  \ref{lma_223} and \ref{prp_224}.)
\end{exc}
\begin{sltn}
  In order for us to show this we must first prove some auxilliary results.
  Firstly, we need to show that multiplication commute with $0$.
  We do this by showing $n \times 0 = 0 \times n$ for all natural numbers $n$.
  Secondly, we need to prove that multiplication commutes with successors.
  That is, $n \times \left( m\dplus \right) = (n \times m) + n$.
  \begin{enumerate}
    \item For any natural number $n, n \times 0 = 0 \times n$.\\
      [0.2cm]
      Inducting on $n$ gives us the base case $0 \times 0 = 0 \times 0$. It is
      clear, that by definition of multiplication both sides equal $0$.

      Now, for the inductive step, assume that it is shown that $n \times 0 = 0
      \times n$.  We now want to show that this holds for the successor
      $n\dplus$.

      Evaluating the left hand side gives us; $(n\dplus) \times 0 = (n \times
      0) + 0$ by definition.  By assumption this equals $(0 \times n) + 0 = 0 +
      0 = 0$.  Right hand side gives $0$ by definition of multiplication.  This
      closes the induction.

    \item For any natural number $n$ and $m$, $n\times(m\dplus) = (n\times m) + n$\\
      [0.2cm]
      We induct on $n$ keeping $m$ fixed. This yields the base case
      $0\times(m\dplus) = (0 \times m) + 0$.
      By definition of multiplication, both the left and right hand side is equal to zero.

      For our inductive step we assume that it is already shown that
      $n\times(m\dplus) = (n\times m) + n$.  We now want to show that
      $(n\dplus)\times(m\dplus) = (n\dplus)\times m) + (n\dplus)$.  Evaluating the
      left hand side, we see that $(n\dplus) \times (m\dplus)$ is by definition
      equal to $n \times (m \dplus) + (m\dplus)$. By our assumption, we know this
      can be rewritten as $(n\times m) + n + (m\dplus)$. By the definition of
      addition this is equal to $(n\times m) + (n+m)\dplus$.

      Right hand side is by definition equal to $(n\times m) + m + (n\dplus)$.
      Definition of addition gives ut that this is equal to $(n\times m) + (m +
      n)\dplus$.  Finally, the commutativity of addition sets this equal to
      $(n\times m) + (n + m)\dplus$.  The two sides are equal. This closes the
      induction.
  \end{enumerate}
  
  Equipped with these two auxilliary results, we can now show that
  multiplication is commutative.  In order to prove that $m \times n = n \times
  m$ we induct on $n$ keeping $m$ fixed.  The base case is $ m \times 0 = 0
  \times m$. Left hand side equates to zero by our first auxilliary result. The
  right hand side equates to zero by definition.

  We now assume that $m \times n = n \times m$. We want to show
  $m\times(n\dplus) = n \times (m\dplus)$.  The left hand side, by the second
  auxilliary result: $m \times (n\dplus) = (m\times n) + m$.  The right hand
  side is by definition equal to $(n \times m) + m$ and applying our assumption
  we have $(m \times n) + m$. The two sides are equal, and this closes the
  induction.
\end{sltn}

\begin{exc}
  Prove Lemma \ref{lma_233}. (Hint: Prove the second statement first) 
\end{exc}

\begin{sltn}

  We first want to prove that if $n, m$ are two positive numbers, then $nm$ is
  also positive.
  
  We induct on $n$ keeping $m$ fixed.  The base case is $n = 0\dplus$.  By
  definition of multiplication we have $n \times m = (0\dplus) \times m =
  (0\times m) + m = m$.  Since $m$ is a positive number the base case holds.

  Now assume that $nm$ is positive.  We now want to show $(n\dplus)m$ is
  positive.  By definition of multiplication, we have $(n\dplus)m = (n\times m)
  + m$. By our assumption, we know that $n\times m$ is a positive number.
  Proposition \ref{prp_228} tells us that the sum of two positive numbers is
  also positive. Therefore $nm$ is positive. 

  We now want to show that $n\times m = 0$ if and only if either one or both of
  $n, m$ is $0$. We show the right implication first.
  
  Assume $n\times m = 0$. If we let be $n$ and $m$ are both positive numbers,
  our previous result tells us that $n \times m$ is positive. By the definition
  of positive, $n \times m \neq 0$ which is a contradiction. Therefore either
  $n$, $m$ or both are zero.
  
  From the definition of multiplication, if we have either $m$, $n$ or both
  zero then $n\times m$ is zero. Our previous result tells us that if both $m$
  and $n$ are positive, i.e., non-zero. Then the product $mn$ is positive.
  Therefore, both implications hold. 

\end{sltn}

\begin{exc}
  Prove Proposition \ref{prp_235}. (Hint: modify the proof of Proposition
  \ref{prp_225} and use the distributive law.) 
\end{exc}

\begin{sltn}
  We want to prove that multiplication is associative.
  That is for any natural numbers $a, b, c$ we have $(a\times b)\times c = a \times (b \times c)$.
  We induct on $c$ keeping $a$ and $b$ fixed.

  The base case where $c = 0$ gives us $0$ on both the left and right hand side
  because of commutativity and the definition of multiplication.

  Assume now that we have shown $(a\times b)\times c = a \times (b \times c)$. We now want to show
  $(a\times b)\times (c\dplus) = a \times (b \times (c\dplus))$

  The left hand, due to commutativity and definition of multiplication is equal
  to $c(ab) + ab$.  This equals $(ab)c + ab$. The right hand side equals, by
  definition and the distributive law, $a(cb + b) = a(cb) + ab = a(bc) + ab = (ab)c + ab$. The two sides
  are equal and this closes the induction.  

\end{sltn}

\begin{exc}
  Prove the identity $\left( a + b \right)^{2} = a^{2} + 2ab + b^{2}$ for all
  natural numbers $a, b$. 
\end{exc}
\begin{sltn}
  We induct on $a$ keeping $b$ fixed.  The base case where $a = 0$ gives for
  the left hand side $(0 + b)^{2} = b^{2}$.  The right hand side equates to, by
  definition of exponentiation $0\times 0^{1} + 0\times(2b) + b^{2} = b^2$.
  
  Now assume that the above identity holds for $a, b$. We want to show that it holds for
  $a\dplus, b$.
  That is $\left( (a\dplus) + b  \right)^{2} = (a\dplus)^{2} + 2(a\dplus)b + b^{2}$.
  
  Equating the left side yields $a^{2} + 2ab + b^{2} + 2b + (2a)\dplus$.
  Equating the right side yields the same $a^{2} + 2ab + b^2 + 2b + (2a)\dplus$.
  
  The calculations involved are quite long, so they are left as an exercise to
  the reader. (First time I'm not on the recieving end of this statement.)
\end{sltn}

\begin{exc}
  Prove Proposition \ref{prp_239}. (Hint: fix $q$ and induct on $n$.)
\end{exc}
\begin{sltn}
  We want to show that there exists natural numbers $m, r$ such that $0 \leq r
  < q$ and $n = mq + r$, where $n$ is a natural number and $q$ a positive
  number.  We induct on $n$ keeping $q$ fixed.

  This yields the base case $n = mq + r = 0$. Chosing $m = r = 0$ this clearly
  holds, even though $q$ is still a positive number. $0 \leq 0 < q$

  We now assume that the statement holds for the natural number $n$. We now
  want to show it must hold for $n\dplus$.

  We have two cases, where $r\dplus < q$ and where $r\dplus = q$.  In the first
  case, where $r\dplus < q$ we can just set $n\dplus = (mq + r)\dplus = (mq +
  (r\dplus))$.

  We now just have to show the second case.\todo{Show the second case}
\end{sltn}

\chapter{Set theory}

In this chapter the more elementary aspects of axiomatic set theory is
presented. Some of the more advanced concepts will be left for later chapters,
but the finer subtetlies of set theory will be left out. 

\section{Fundamentals}

We start with an informal definition of what a set \textit{should} be.

\begin{defn}[Informal]
  \label{defn_311}
  We define a \textit{set} $A$ to be any unordered collection of objects, e.g.,
  $\left\{ 3, 8, 5, 2 \right\}$ is a set. If $x$ is an object, we say that $x$
  \textit{is an element of} $A$ or $x \in A$ if $x$ lies in the collection;
  otherwise we say that $x \notin A$. For instance, $3 \in \left\{ 1, 2, 3, 4,
  5 \right\}$ but $7 \notin \left\{ 1, 2, 3, 4, 5 \right\}$.
\end{defn}

We first want to clarify the fact that sets themselves are considered objects.
Therefore, we impose the following axiom:

\begin{axm}[Sets are objects]
  \label{axm_31}   
  If $A$ is a set, then $A$ is also an object. In particular, given two sets
  $A$ and $B$, it is meaningful to ask whether $A$ is also an element of $B$.
\end{axm}

An example of sets being elements of other sets would be the set $\left\{ 3,
\left\{ 3, 4 \right\}, 4 \right\}$.  It consists of three distinct objects, one
which happens to also be a set. Not all objects are sets. A natural number is
not typically considered a set.\footnote{A natural number can be the
\textit{cardinality} of a set however.}

More specifically, if $x$ is an object and $A$ is a set, then either $x \in A$
is true, or $x \in A$ is false. If $A$ is not a set, then the statement $x \in
A$ is neither true nor false but meaningless.

\addtocounter{dummy}{2}

We now define the notion of equality between sets.
\begin{defn}[Equality of sets]
  \label{defn_314}
  Two sets $A$ and $B$ are \textit{equal}, $A = B$ if and only if for every
  element of $A$ is an element of $B$ and vice versa. To put it another way, $A
  = B$ if and only if every element $x$ of $A$ belongs also to $B$, and every
  element $y$ of $B$ belongs also to $A$.
\end{defn}

\addtocounter{dummy}{1}

A neat little observation is that if $x \in A$ and $A = B$, then $x \in B$ by
Definition \ref{defn_314}. Therefore, the ``is an element of'' relation $\in$
obeys the axiom of substitution. That is, you can substitute $B$ for $A$ in the
statement $x \in A$. This will, as we shall see, be the case for the remaining
defintions in this section.

Now that we have defined the notion of a set, we want do discern which objects
can be considered sets and which objects cannot. Starting with a single set, the \textit{empty set}.

\begin{axm}[Empty set]
  \label{axm_32}
  There exists a set $\emptyset$, known as the empty set, which contains no
  elements, i.e., for every object $x$ we have $x \notin \emptyset$. 
\end{axm}

One can easily prove the uniqueness of the empty set by assuming that there are
two empty sets and showing that they must be equal by Definition
\ref{defn_314}. We now examine what it means for a set to be
\textit{non-empty}.

\begin{lma}[Single choice]
  \label{lma_316}
  Let $A$ be a non-empty set. Then there exists an object $x$ such that $x \in
  A$.  
\end{lma}

\begin{proof}
  Assume for contradiction that $A$ is non-empty, and there exist no objects
  $x$ such that $x \in A$.  We see that our assumption about no objects $x \in
  A$ coincides with Axiom \ref{axm_32}.  By definition
  \ref{defn_314}, we must have $A = \emptyset$. However, this contradicts our
  assumption about $A$ being non-empty. Therefore there must exist an object
  $x$ such that $x \in A$. 
\end{proof}

If Axiom \ref{axm_32} was the only axiom set theory had, then there might be
just a single set in existence, namely the empty set. The following axioms are
here to enrich the number of sets we will have available to us. 

\begin{axm}[Singleton sets and pair sets]
  \label{axm_33}
  If $a$ is an object, then there exists a set $\left\{ a \right\}$ whose only
  element is $a$, i.e., for every object $y$, we have $y \in \left\{ a
  \right\}$ if and only if $y = a$; we refer to $\left\{ a \right\}$ as the
  \textrm{singleton set} whose element is $a$.  Furthermore, if $a$ and $b$ are
  objects, then there exists a set $\left\{ a, b \right\}$ whose only elements
  are $a$ and $b$; i.e., for every object $y$, we have $y \in \left\{ a, b
  \right\}$ if and only if $y = a$ or $y = b$; we refer to this set as the
  \textrm{pair set} formed by $a$ and $b$. 
\end{axm}
\addtocounter{dummy}{4}

It is important to note that there is only one singleton set for each object
$a$. This follows from definition \ref{defn_314}. Similarly, there is only one
pair set formed by two objects $a$ and $b$. Definition \ref{defn_314} actually
ensures that $\left\{ a, b \right\} = \left\{ b, a \right\}$ and that $\left\{
a, a \right\} = \left\{ a \right\}$. One could keep assuming new axioms for larger and larger sets, with more and
more elements, but it is shown that it suffices with the previous two once we assume the next axiom, namely;
\begin{axm}[Pairwise union]
  \label{axm_34}
  Given any two sets $A$, $B$, there exists a set $A \union B$, called the \textit{union} $A \union B$ of $A$ and $B$, whose elements
  consists of all the elements which belong to $A$ or $B$ or both. In other words, for any object $x$,
  \begin{equation}
    \notag
    x \in A \union B \Longleftrightarrow \left( x \in A \text{ or } x \in B\right).
  \end{equation}
\end{axm}
\addtocounter{dummy}{2}
We now show some basic properties of unions.

\begin{lma}
  \label{lma_3113} 
  If $a$ and $b$ are objects, then $\left\{ a, b \right\} = \left\{ a \right\}
  \union \left\{ b \right\}$. If $A, B, C$ are sets, then the union operation
  is commutative (i.e., $A \union B = B \union A$) and associative (i.e.,
  $\left( A\union B \right)\union C = A\union(B\union C)$). Also, we have $A
  \union A = A \union \emptyset = \emptyset \union A = A$. 
\end{lma}

\begin{proof}
  We prove the associativity identity here and leave the rest for Exercise 3.1.3.

  We want to show that $\left( A\union B \right)\union C = A\union(B\union C)$
  By definition \ref{defn_314} we need to show that every element of $\left(
  A\union B \right)\union C$ is also an element of $A\union(B\union C)$, and
  vice versa.

  Suppose first that $x$ is an element of $(A \union B) \union C$. By Axiom
  \ref{axm_34} we have that either $x \in (A \union B)$ or $x \in C$. In other words, at
  least one of the two statements must be true.
  We now divide the proof into two cases:
  
  \begin{enumerate}
    \item $ x \in C$ \\
      [0.2cm]
      Applying Axiom \ref{axm_34} several times tells us that if $x \in C$,
      then $x \in (B \union C)$, and finally $x \in A \union (B \union C)$.

    \item $ x \in (A \union B)$ \\
      [0.2cm]
      Again applying Axiom \ref{axm_34} several times gives that if $x \in (A
      \union B)$ then $x \in A$ or $x \in B$.  If $x \in A$, then $x \in
      A\union(B\union C)$, and if $x \in B$ then $x \in (B \union C)$, which in
      turn means that $x \in A \union (B \union C)$. 
  \end{enumerate}

  A very similar argument shows that if $x \in A \union (B \union C)$, then $ x
  \in (A\union B) \union C$.  Therefore $\left( A\union B \right)\union C = A\union(B\union C)$.
\end{proof}
\todos
\end{document}

