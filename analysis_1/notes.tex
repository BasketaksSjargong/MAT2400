%        File: notes.tex
%     Created: Tue Mar 24 01:00 PM 2015 C
% Last Change: Tue Mar 24 01:00 PM 2015 C
%
\documentclass[a4paper, twocolumn]{report}

\usepackage[]{amsthm} 
\usepackage[]{amsmath} 
\usepackage[]{amssymb} 
\usepackage[draft]{todonotes} 
\setlength{\marginparwidth}{2cm}

\newcounter{dummy} \numberwithin{dummy}{section}
\newcounter{axmcntr} \numberwithin{axmcntr}{chapter}
\newtheorem{defn}[dummy]{Definition}
\newtheorem{axm}[axmcntr]{Axiom}
\newtheorem{prp}[dummy]{Proposition}
\newtheorem{asm}[dummy]{Assumption}
\newtheorem{lma}[dummy]{Lemma}
\newtheorem{crl}[dummy]{Corollary}
\newtheorem*{crl*}{Corollary}

\newcommand{\dplus}{{+}{+}}


\title{
  \textsc{University of Oslo} \\
  \textsc{Real Analysis - MAT2400} \\
  \textsc{Spring 2015}
}
\author{Ivar Stangeby}
\date{\today}
\begin{document}
\twocolumn[
\begin{@twocolumnfalse}
\maketitle
\begin{abstract}
This document is going to be a way for me as a student of MAT2400 at the
University of Oslo to gather my thoughts around the course Real Analysis.  I've
struggled with my intuition for this subject, and this is a last ditch effort
to build it all up from scratch.

For this task, I've decided to use the text books written by Terence Tao,
namely Analysis I and II. There is absolutely nothing wrong with the text book
offered at my university, it is brilliantly written, but I have been exposed to
the writings of Terence Tao before, and therefore I wish to give his books a
try.

The first part of this document related to the book Analysis I, is going to be
mostly involved in building a solid foundation for the concepts discussed in
Analysis II. The material included from Analysis I is very similar to the
curriculum of the subject MAT1140 at the University of Oslo.

The structure of this document is going to be me writing down the results
encountered throughout the text books along the proofs I find extra intriguing.
I'm going to attempt to prove the theorems myself, and if I find it reasonable
I'm going to write down my own proof. Included will also be my attempted
solutions to selected exercises. 

This document is mainly for my own good and well being, but if anyone can find
any use from them, then that is great.

\end{abstract}
\end{@twocolumnfalse}
]
\chapter{Introduction}

\chapter{Starting at the beginning: the natural numbers.}
\label{ch:natural_numbers}

In order for us to start exploring the various properties of the real numbers,
which is what real analysis is concerned with, we are going to have to start
from the very beginning.  That is the natural numbers, denoted $\mathbb{N}$.
From these natural numbers, we can construct the integers, $\mathbb{Z}$, the
rationals $\mathbb{Q}$, the real numbers $\mathbb{R}$, and finally; the complex
numbers $\mathbb{C}$. The latter being the main focus of the subject Complex
Analysis.

\todo{Replace all occurances of $++$ with a proper unary plus sign. To remove wonky spacing.}
\section{The Peano axioms}
One of the most standard ways of defining the natural numbers, is in terms of
the \textit{Peano axioms}. One can also define natural numbers through the
notion of cardinality.

\begin{defn}[Informal]
  A \textit{natural number} is any element of the set
  \begin{equation}
    \notag
    \mathbb{N} = \left\{ 0, 1, 2, 3, 4, \cdots \right\},
  \end{equation}
  which is the set of all the numbers created by starting with 0 and then
  counting forward indefinitely. We call $\mathbb{N}$ the \textit{set of all
  natural numbers}.
\end{defn}

In order for us to rigorously define the set of natural numbers, we're going to
use the two fundamental concepts of \textit{the number 0} and the
\textit{increment operation}.  These will be covered in the Peano Axioms.
We will use $n++$ to denote the \textit{successor} of $n$. 

Starting with the first two: 
\begin{axm}
  0 is a natural number.
\end{axm}
\begin{axm}
  If $n$ is a natural number, then $n++$ is also a natural number.
\end{axm}

Now, in order to avoid having to deal with incredibly long strings of $+$'es.
We're going to use an auxilliary definition.
\begin{defn}
  We define 1 to be the number $0++$, 2 to be the number $(0++)++$, etc.
\end{defn}
We can based off of this, propse the following:

\begin{prp}
  3 is a natural number.
\end{prp}
\begin{proof}
  By Axiom 1, 0 is a natural number.
  It then follows by Axiom 2 that both 1, 2, and 3 are natural numbers.
\end{proof}

In order for us to avoid the problem of having the successive numbers wrap
around to previous numbers, we impose a new axiom, namely:

\begin{axm}
  0 is not the successor of any natural number: i.e., we have $n++ \neq 0$ for
  every natural number $n$.
\end{axm}

We can, equipped with this new axiom, show for example the following:

\begin{prp}
  $4$ is not equal to $0$.
\end{prp}
\begin{proof}
  By definition, $4 = 3++$. By the first two axioms, $3$ is a natural number.
  Thus, since $0$ is not the successor of any natural number, $3++ \neq 0$,
  i.e., $4 \neq 0$. 
\end{proof}

Assuming the following axiom allows us to rule out any behaviour where the
successors wrap around, but not to $0$, i.e., $5++ = 1$.

\begin{axm}
  Different natural numbers must have different successors; i.e., if $n, m$ are
  natural numbers and $n \neq m$, then $n++ \neq m++$. Equivalently, if $n++ =
  m++$, then we must have $n = m$.
\end{axm}

We can now prove extensions of the previous proposition where we do not have zeroes
on the right hand side of the equation.

\begin{prp}
  $6$ is not equal to $2$.
\end{prp}
\begin{proof}
  Assume for contradiction that $6 = 2$. By the previous axiom we must have
  $5++ = 1++$.  Applying the same axiom again, we have $5 = 1$ so that $4++ =
  0++$. But, this leads to a contradiction, because by the same axiom, $4 = 0$.
  This contradicts our previously proven proposition.
\end{proof}

Assume now that we are presented with a weird number system
\begin{equation}
  \notag
  \mathbb{N} = \left\{ 0, 0.5, 1, 1.5, 2, 2.5, 3, 3.5, \dots \right\}.
\end{equation}
Even though this set contains real numbers, which we haven't defined or talked
about yet, it satisfies all the previous axioms. But this is not the number
system we're interested in. We want our set of natural numbers to only be
containing all the numbers that can be directly derived from $0$ just using the
successor operation.

We want to introduce some axiom that does not allow other forms of successors to occur.
Therefore we introduce the following:

\todo{Strictly speaking, this axiom is not an axiom but an \textit{axiom schema}.}
\begin{axm}[Principle of mathematical induction]
  Let $P\left( n \right)$ be any property pertaining to a natural number $n$.
  Suppose that $P\left( 0 \right)$ is true, and suppose that whenever $P\left(
  n \right)$ is true, $P(n++)$ is also true. Then $P(n)$ is true for every
  natural number $n$. 
\end{axm}

We're now equipped with the tools required to deal with propositions of the following form:
\begin{prp}
  A certain property $P(n)$ is true for every natural number $n$.  
\end{prp}
\begin{proof}
  Using induction, we show the base case of $P(0)$. Assume, for the sake of
  induction, that $P(n)$ is true. We now want to show that it has to follow
  that $P(n++)$ also must be true. If this is the case, we have shown, using
  mathematical induction that $P(n)$ is true for every natural number $n$.
\end{proof}

The previous five axioms are known as the \textit{Peano Axioms} for the natural
numbers.  We now want to more rigorously define the kind of number system we
are to refer to as the \textit{natural numbers}.

\begin{asm}[Informal]
  There exists a number system $\mathbb{N}$, whose elements we will call
  \textrm{natural numbers}, for which Axioms 1-5, are true.
\end{asm}

This number system is what we refer to as \textit{the} natural number system.
But one should not rule out the possibility that there are more than one
natural number system. But as long as these are \textit{isomorphic} one can
consider them as equal.

With only this, rather simplistic definition of natural numbers, the five
axioms, some axioms from set theory we can build all other number systems,
create functions and do the algebra and calculus that we are used to.

An very common question now arises, and this is about the finiteness or
infiniteness of the natural number system.  How can something infinite come
from something strictly finite?  One can easily show that all the natural
numbers are finite. It is clear that $0$ is finite. If $n$ is finite, then
clearly $n++$ is finite. Therefore all natural numbers are finite.  It then
follows that infinity is not a natural numbers. There are other number systems
that admit the infinite numbers.  

It is an interesting fact that the definition of $\mathbb{N}$ is
\textit{axiomatic} rather than \textit{constructive}. This means, that so far
we're only conserned with what the natural numbers are, not what they do, what
they measure or what they can be used for.

As long as a mathematical model obeys the previous axioms, it is of no concern
whether which mathematical model is ``true''. It is this form of
\textit{abstractness} that makes mathematics so useful. One does not
neccesarily need a concrete model, because the numbers can be understood
abstractly through the use of axioms.

As a consequence of the axioms previously discussed, we can now define
sequences \textit{recursively}.  That is, start with some base value and then
building the next value in the sequence by means of a function.  This leads to
the following:

\begin{prp}[Recursive definitions]
  Suppose for each natural number $n$, we have some function $f_n:\mathbb{N}
  \rightarrow \mathbb{N}$ from the natural numbers to the natural numbers. Let
  $c$ be a natural number.  Then we can assign a unique natural number $a_n$ to
  each natural number $n$, such that $a_0 = c$ and $a_{n++} = f_{n}\left( a_n
  \right)$ for each natural number $n$. 
\end{prp}
\begin{proof}
  Using induction, we verify the base case. We clearly see that this procedure
  gives a single value to $a_0$, namely $c$. (We know from axiom 3 that $a_0$
  won't be redefined.) Suppose now inductively that the procedure gives a
  single value to $a_n$. Then it gives a single value to $a_{n++}$, namely
  $a_{n++} = f_n\left( a_n \right)$. (We know from axiom 4 that $a_{n++}$ won't
  be redefined.) This completes the induction, since $a_n$ is defined for every
  natural number $n$, with a single value assigned to each $a_n$.
\end{proof}

Equipped with the tools that are recursive definitions we can now define
multiple operations on the set of natural numbers. Up until now, we've only
dealt with one, being the increment operation.

\section{Addition}

Currently our number system does not support any more advanced operations than
incrementing a number.  We now turn our heads to addition. The operation is
simple. To add $3$ to $5$, we simply increment $5$ three times. This is one
increment more than adding $2$ to $5$, which is one increment more than adding
$1$ to $5$, which is one increment more than adding $0$ to $5$. We can
therefore easily give a recursive definition of addition.

\begin{defn}[Addition of natural numbers]
  Let $m$ be a natural number. To add zero to $m$, we define $0 + m = m$. Now
  suppose inductively that we have defined how to add $n$ to $m$. Then we can
  add $n++$ to $m$ by defining $(n++) + m = (n + m)++$.        
\end{defn}

For example, $2 + 3 = (3++)++ = 4++ = 5$. By using the priciple of mathematical
induction, we see that we now have defined $n + m$ for every natural number
$n$. We are specializing the previous general discussion about recursive
definitions to the setting where $a_{n} = n + m$, and $f_{n}\left( a_n \right)
= a_{n}++$.

It's worth noting that this definition is actually \textit{asymmetric}. That
is, while yielding the same result, $3 + 5$ is incrementing $5$ three times,
where as $5 + 3$ is incrementing $3$ five times. We shall soon see, that it is
a general fact that $a + b = b + a$ for all natural numbers $a, b$. 

One can easily prove, using the first two axioms and the principle of
mathematical induction to show that the sum of two natural numbers us again a
natural number.

At the present moment, we have only two facts about addition. However, this is
perfectly sufficient to deduce everything else we know about addition. Starting
with some basic lemmas.

\begin{lma}
  For any natural number $n$, $n + 0 = n$. 
\end{lma}

This lemma is not obvious from our previous definition, since we still do not
know that $a + b = b + a$. 

\begin{proof}
  Using induction. The base case $0 + 0 = 0$ follows from the definition of
  addition of natural numbers.  $0 + m = m$ for all natural numbers, and $0$ is
  known to be a natural number.  Suppose inductively that $n + 0 = n$. We now
  wish to show that $(n++) + 0 = n++$. By definition of addition yields that
  $(n++) + 0$ is equal to $(n + 0)++$, which is equal to $n++$ since $n + 0 =
  n$. This closes the induction.
\end{proof}

\begin{lma}
  For any natural numbers $n$ and $m$, $n + (m++) = (n + m)++$.
\end{lma}

Again, this is not obvious. 

\begin{proof}
  We induct on $n$, keeping $m$ fixed. Considering the base case $n = 0$. We
  therefore have to prove $ 0 + (m\dplus) = (0 + m)\dplus$. By definition of
  addition, we have $0 + (m\dplus) = m\dplus$ and $0 + m = m$. So, both sides
  are equal to $m\dplus$ and are thus equal. Assuming now, that we have shown
  $n + (m\dplus) = (n + m)\dplus$, we want to show that $(n\dplus) + (m\dplus)
  = \left( \left( n\dplus \right) + m \right)\dplus$.  Looking at the left hand
  side. By definition of addition it is equal to $\left( n + \left( m\dplus
  \right) \right)\dplus$, which in turn, by the inductive hypothesis, is equal
  to $\left(\left( n + m \right)\dplus\right)\dplus$. Now, examining the right
  hand side. By the definition of addition, it is equal to $\left(\left( n + m
  \right)\dplus\right)\dplus$. The two sides are equal, and this closes the
  induction.
\end{proof}
We can now easily show the following result:
\begin{crl*}
  For all natural numbers $n$, $n\dplus = n + 1$.    
\end{crl*}
\begin{proof}
  This is a special case of the previous lemma, where $m = 0$.  We have $n +
  (m\dplus) = (n + m)\dplus$. Setting $m = 0$, we get $n + (0\dplus) = (n +
  0)\dplus$. Evaluating the left hand side we get $n + 1$ and evaluating the
  right hand side, we get $(n)\dplus = n\dplus$, which is what we wanted to
  show.
\end{proof}

Now for one of the first major results. We can now prove that $a + b = b + a$.

\begin{prp}[Addition is commutative]
  For any natural numbers $n$ and $m$, $n + m = m + n$. 
\end{prp}
\begin{proof}
  Using induction on $n$ keeping $m$ fixed.  First showing the base case, where
  $n = 0$.  We want to show that $0 + m = m + 0$.  By the definition of
  addition, the left hand side is equal to $m$. By lemma 2.2.2, the right hand
  side is equal to $m$. Therefore, the base case is true.  Now, assuming that
  it is shown that $n + m = m + n$.  We now want to show that $(n\dplus) + m =
  m + (n\dplus)$.  Looking at the left hand side, we see that it is equal to
  $(n + m)\dplus$, by definition of addition.  The right hand side, by lemma
  2.2.3 must be equal to $(m + n)\dplus$. Since we assumed $n + m = m + n$, the
  left and right hand side is equal and therefore the induction is closed.
\end{proof}

\begin{prp}[Addition is associative]
  For any natural numbers $a, b, c$, we have $(a + b) + c = a + (b + c)$.  
\end{prp}
\begin{proof}
  See Exercise 2.2.1.
\end{proof}

\begin{prp}[Cancellation law]
  Let $a, b, c$ be natural numbers such that $a + b = a + c$. 
\end{prp}

Since we haven't explored the concept of subtraction or negative numbers yet we
cannot use these properties to prove this law. This law is actually crucial in
defining subtraction rigorously. 

\begin{proof}
  We prove this with induction on $a$, keeping $b$ and $c$ fixed.  Showing the
  base case with $a = 0$. We have $0 + b = 0 + c$. Using the definition of
  addition, $b = c$. Now, for the inductive hypothesis.  Assuming that it is
  shown that $ a + b  = a + c$, we want to show that $(a\dplus) + b = (a\dplus)
  + c$ implies $b = c$.  Left hand side evaluates to $(a+b)\dplus$ and the
  right hand side evaluates to $(a + c)\dplus$ by the definition of addition.
  By Axiom 2.4 we see that $(a + b) = (a + c)$, and therefore by our assumption
  $b = c$. This closes the induction.
\end{proof}

We now want to look at how natural numbers interacts with positivity. First, a definition:
\begin{defn}[Positive natural numbers]
  A natural number $n$ is said to be \textit{positive} if and only if it is not equal to $0$.
\end{defn}

This leads to the following proposition.
\begin{prp}
  If $a$ is positive and $b$ is a natural number, then $a + b$ is positive (and hence $b + a$ is also, by Proposition 2.2.4).
\end{prp}
\begin{proof}
  Using induction on $b$. Showing the base case where $b = 0$.  Assuming $a$ a
  positive number and $b$ a natural number.  We then have $ a + b = a + 0 = a$,
  and by assumption $a$ is a positive number.  Now for the inductive step. We
  assume shown that $a + b$ is a positive number.  We want to show that $a + (
  b\dplus)$ must also be a positive number. Using the commutativity of natural
  numbers and the definition of addition we can show that this must be equal to
  $(a + b)\dplus$. Since we know that $0$ is not the successor to any number,
  and that $(a + b)$ is positive, we must have $(a + b)\dplus \neq 0$. This
  closes the induction.
\end{proof}

\begin{crl}
  If $a$ and $b$ are natural numbers such that $a + b = 0$, then $a = 0$ and $b = 0$. 
\end{crl}


\end{document}



