\documentclass{homework}
\author{Ivar Stangeby}
\title{Real Analysis --- Exercises}
\date{Spring 2016}
\course{MAT2400}

\begin{document}
\week{1} 

\problem{1.1.1}
\begin{problemtext}
Assume that the product of two integers $x$ and $y$ is even. Show that at least
one of the numbers is even.
\end{problemtext} 
\begin{solution}
We want to prove the implication $xy$ even $\Longrightarrow$ $x$ even or $y$
even. We prove this by contraposition. Assume that $x$ and $y$ are both odd.
Then for some $n, m \in \mathbb{N}$ we have $x = 2n + 1$ and $y = 2m+1$.  It
then follows that $xy = \left(2n + 1\right)\left(2m + 1\right) = 4nm + 2(n + m)
+ 1 = 2(2nm + (n + m)) + 1$. Hence $xy$ is odd. By the contrapositive proof our
original implication holds.
\end{solution}

\problem{1.1.2}
\begin{problemtext}
   Assume that the sum of two integers  $x$ and $y$ is even. Show that $x$ and
   $y$ are either both even or both odd.
\end{problemtext}
\begin{solution}
    Again, we proceed by contrapositive. Assume that $x$ is even and $y$ is odd
    (the other case follows by symmetry). We then have that $x +  y = 2n + 2m +
    1$ for some $n, m \in \mathbb{N}$. Hence $x + y$ is odd. We have therefore
    proved the contrapositive statement, so the original implication holds.
\end{solution}

\problem{1.1.3}
\begin{problemtext}
    Show that if $n$ is a natural number such that $n^2$ is divisible by 3,
    then $n$ is divisible by 3. Use this to show that $\sqrt{3}$ is irrational.
\end{problemtext}
\begin{solution}
    Assume that $n$ is not divisible by $3$. This means that $n = 3m + r$ for
    some integer $0 < r < 3$.  Then $n^2 = (3m + r)^2 = 9m + 6mr + r^2 = 3m(3 +
    2r) + r^2$. This is only divisible by $3$ is $r^2$ is divisible by $3$, but
    $r = 1$ or $r = 2$ are the only two cases we have, hence $r^2 = 1$ or $r^2
    = 4$, with neither being divisible by 3. In other words $n^2$ is not
    divisible by 3. This concludes the proof.

    We now want to show that $\sqrt{3}$ is irrational. We assume for
    contradiction that it is rational. Let $\sqrt{3} = m / n$. Also assume that
    $m, n$ have no common factors. Then $3 = m^2 / n^2 = q$. Since $q$ is
    divisible by $3$ we have that $\sqrt{3}$ is divisible by 3. So $\sqrt{3} =
    3p$ for some $p \in \mathbb{N}$. 
\end{solution}
\end{document}
