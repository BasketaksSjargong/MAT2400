%        File: solution.tex
%     Created: Tue Feb 24 01:00 PM 2015 C
% Last Change: Tue Feb 24 01:00 PM 2015 C
%
\documentclass[a4paper]{article}
\usepackage{amsmath}
\usepackage{amsthm}
\usepackage{amsfonts}
\newtheorem{prb}{Problem}
\newtheorem{defn}{Definition}
\theoremstyle{plain}
\newtheorem{thm}{Theorem}
\newtheorem{prp}{Proposition}
\title{
  \huge MAT2400\\
  \centering{Assignment 1}
  \author{Ivar Stangeby}
  \date{\today}
}
\begin{document}
\maketitle
\begin{prb}
  Show that a strictly increasing function $f: \mathbb{N} \rightarrow
  \mathbb{N}$ must satisfy $f(n) \geq n, \forall n \in \mathbb{N}$.
\end{prb}

\paragraph{Solution: }
Assume that $f: \mathbb{N} \rightarrow \mathbb{N}$ is a strictly increasing
function.  By the definition of a strictly increasing function we know that
\begin{equation}
  \notag
  f(n + 1) > f(n), \forall n \in \mathbb{N}.
\end{equation}

We can now, since we are working with the natural numbers, easily show this
inductively.  Let us first show the base case.
\begin{align*}
  \label{eq:}
  f(1) \geq 1  
\end{align*}

This is intuitively true, because $1$ is defined as the least element of the
set of natural numbers.  Assuming that we have verified this as true for all
$n$ up to and including some number $k$.  We know want to show that it then
follows that it must be true for $k + 1$. 
By assumption:
\begin{equation}
  \notag
  f(k) \geq k
\end{equation}
Using the standard metric in $\mathbb{R}$ we can see that for any two pairs of successive
integer numbers, 
\begin{align*}
  \inf\left\{ d(k, k+1) \mid k \in \mathbb{N} \right \} = 1 \\
  \intertext{ where,}
  d(x, y) = |x - y|
\end{align*}

That is, the smallest distance possible with two different numbers is 1.  It
then follows that
\begin{align*}
  \label{eq:}
  f(k+1) > f(k) + 1 \geq k+1 \\
  f(k+1) \geq k+1
\end{align*}
as we wanted to show.  Thus, by the induction principle, a strictly increasing
function from $\mathbb{N}$ to $\mathbb{N}$, must neccessary satisfy $f(n) \geq
n, \forall n \in \mathbb{N}$.

\begin{prb}
Let $\left( X, d) \right)$ be a complete metric space. Let $B(x, r)$ denote the open ball
centered at $x \in X$ with radius $r$, i.e., 
\begin{equation}
  \notag
  B(x, r) = \left\{ y \in X \mid d(x, y) < r \right\},
\end{equation}
and $\overline{B}\left( x, r \right)$ the closed ball of radius $r$, i.e.,
\begin{equation}
  \notag
  \overline{B}\left( x, r \right) = \left\{ y \in X \mid d(x, y) \leq r \right\}.
\end{equation}
For any set $C \in X$, let $\overline{C}$ denote its closure.
Is it true that for any complete metric space $X$, 
\begin{equation}
  \label{eq:condition}
  \overline{B\left( x, r \right)} = \overline{B}\left( x, r \right)?
\end{equation}
\end{prb}

\paragraph{Solution: }

Consider the discrete metric,
\begin{equation}
  \notag
  d(x, y) = \begin{cases}
    0 & \text{if } x = y \\
    1 & \text{if } x \neq y
  \end{cases}
\end{equation}

We can show that \eqref{eq:condition} does not neccessarily hold under the
discrete metric.  Lets assume we take the radius $r$ to be $1$.  The open ball
$B(x, r)$ is then any two points with less than a distance $r$ between.  Thus
the open ball only contains the point $x$. In that case, taking the closure of
this open ball changes nothing, and we're left with just the point $x$.
However, the closed ball $\overline{B}\left( x, r \right)$ has to be the
entirety of our space $X$, since the distance between two points are allowed to
be $1$.  Thus, if we let our metric space be $(X, d)$ with $X = \mathbb{R}$ and
$d$ the discrete metric \eqref{eq:condition} does not hold. We then have a
complete metric space $(R, d)$. We then have a complete metric space $(R, d)$.

\begin{prb}
Let $\ell$ be the set of sequences of real numbers where only a finite number of terms
are different from zero,
\begin{equation}
  \notag
  \ell = \left\{ \left\{ x_n \right\}_{n=1}^{\infty} \mid x_i = 0 \text{ for all but a finite number of $i$'s} \right\}.
\end{equation}
For $x = \left\{ x_n \right\}$ and $y = \left\{ y_n \right\}$ in $\ell$, define
\begin{equation}
  \notag 
  d(x, y) = \sup_{n\in\mathbb{N}} |x_n - y_n|.
\end{equation}
\end{prb}

\paragraph{Solution: }
\paragraph{a)}

To show that $d$ is a metric on $\ell$ we must show the three properties of a metric function.
\begin{enumerate}
  \item Positivity:
    Since the metric is defined as the biggest difference between to
    corresponding elemnts from $\left\{ x_n \right\}$ and $\left\{ y_n
    \right\}$, the metric must neccesarily satisfy the property of positivity
    since there does exists a finite number of non-zero elements in each sequence.
    Thus, $d(x, y) \geq 0$ with equality only if $x = y$. 
  \item Symmetry:
    
    \begin{equation}
      \notag
      d(x, y) = \sup_{n\in\mathbb{N}} |x_n - y_n| = \sup_{n\in\mathbb{N}}|y_n - x_n| = d(y, x).
    \end{equation}
    Thus the metric is symmetric.

  \item Triangle Inequality:
    Want to show that given three sequences $x, y, z$, the metric satisfies
    \begin{equation}
      \notag
      d(x, z) \leq d(x, y) + d(y, z).
    \end{equation}
    The trivial case, when $x = y = z$ is just that, trivial. Thus we assume
    that $x, y$ and $z$ are not equal.
        
\end{enumerate}

\paragraph{b)}

Letting $u_k \in \ell$ be defined as
\begin{equation}
  u_k = \left\{ 1, \frac{1}{2}, \dots, \frac{1}{k}, 0, 0, 0, \dots \right\}
\end{equation}
we want to show that $\left\{ u_k \right\}_{k=1}^{\infty}$ is a Cauchy sequence
in $(\ell, d)$. We can do this using a traditional $\varepsilon-N$-proof.  We
want to show that for all $\varepsilon > 0$ there exists an $N \in \mathbb{N}$
such that for all integers $m, n > N, \, d(u_m, u_n) < \varepsilon$.

Assume that $m > n$. We're then going to have two sequences looking like this:
\begin{align*}
  \label{eq:}
  u_n &= \left\{ 1, \frac{1}{2}, \frac{1}{3},  \dots , \frac{1}{n}, 0, 0, 0, \dots \right\} \\
  u_m &= \left\{ 1, \frac{1}{2}, \frac{1}{3},  \dots,  \frac{1}{n}, \dots, \frac{1}{m-1}, \frac{1}{m}, 0, 0, 0, \dots \right\}
\end{align*}
By observation, we see that
\begin{align*}
  d(u_m, u_n) &= \sup_{i\in\mathbb{N}}\left\{ \left|u_{m_{i}} - u_{n_{i}} \right|\right\} \\
  &= \left|u_{m_{n+1}} - u_{n_{n+1}}\right|
\end{align*}
Since given any $\varepsilon > 0$, we can always choose an $N \in \mathbb{N}$
such that we can get numbers on the form $\frac{1}{n}$ as close to $0$ as we
want. Certainly smaller than $\varepsilon$.  Thus it follows that $\left\{ u_k
\right\}_{k=1}^{\infty}$ is a Cauchy sequence in $\left( \ell, d \right).$

\paragraph{c)}

Let $\ell$ be a subset of a metric space $\left( X, d \right)$.  Let $x \in
\ell$, and choose a subsequence $x_s$ of $x$ that only contains all the
zero-elements of $x$. We can do this by shifting sufficiently far to the right
in the sequence, so that all the non-zero elements are to the left.  By
observation we see that $x_s$ must converge to the null sequence $x_0 \in
\ell$.  I've decided to interpret the condition that $\ell$ must contain a
finite number of non-zero elements such that it can also contain zero non-zero
elements. 

Since we can chose such a subsequence for any sequence in $\ell$, the metric
space $\left( \ell, d \right)$ is compact. It then follows, that since every compact metric space is complete, and
in a complete space all Cauchy sequences converge, thus $\left\{ u_k \right\}$ must converge.

\paragraph{d)}

Let $c_0$ be defined as follows:
\begin{equation}
  \notag
  c_{0} = \left\{ \left\{ x_n \right\}_{n=1}^{\infty} \mid \lim\limits_{n\rightarrow\infty} x_{n} = 0 \right\}
\end{equation}
In order to show that $c_0$ is a compact space under the metric $d$ we have to
show that all Cauchy sequences in $c_0$ converge.  

Assume then that $x \in c_{0}$ is Cauchy.
This means that for all $\varepsilon$ there exists an $N \in \mathbb{N}$ such that given $m, n > N, d(x_n, x_m) < \varepsilon$.

\paragraph{e)}

$\ell$ is dense in $c_{0}$ if and only if for each $x \in c_{0}$ there is a
sequence $\left\{ y_n \right\}$ from $\ell$ converging to $x$.  Thus we want to
show that given an $x$ from $c_{0}$, we can always produce a sequence $\left\{
y_{n} \right\}$ from $\ell$ converging to $x$.  Observe that the sequence
$\left\{ 0, 0, 0, \dots \right\}$ from $\ell$ satisfies this for all sequences
$x \in c_{0}$.  If $\left\{ y_n \right\}$ are to converge to $x$ means that

\begin{equation}
  \notag
  \lim\limits_{n\rightarrow \infty} d(y_{n}, x_{n}) = 0
\end{equation}

Since all sequences in $c_{0}$ converge to zero, the distance between elements
in the sequence $y_{n}$ and $x$ must tend to zero. Thus, $\ell$ is dense in
$c_{0}$.

\begin{prb}
  Let $X$ denote the open interval $\left( 0, \infty \right) \subset \mathbb{R}$. Let $d: X \times X \rightarrow \mathbb{R}$ be defined as
  \begin{equation}
    \notag
    d(x, y) = \left|\ln(x) - \ln(y)\right|.
  \end{equation}
\end{prb}

\paragraph{Solution:}
\paragraph{a)}
Again, to show that $d$ is a metric we must show the three properties.
The first two, positivity and symmetry are trivial. We need to show that the triangle inequality holds.
That is,

\begin{align*}
  \label{eq:}
    d(x, z) &= \left|\ln(x) - \ln(z)\right| \\
    &= \left|\ln(x) - \ln(y) + \ln(y) - \ln(z)\right|\\
    &= \left|\ln\left(\frac{x}{y}\right) + \ln\left( \frac{y}{z} \right)\right| \leq \left|\ln\left( \frac{x}{y} \right)\right| + \left|\ln\left( \frac{y}{z} \right)\right| \\
    &= \left|\ln\left( x \right) - \ln\left( y \right)\right| + \left|\ln\left( y \right) - \ln\left( z \right)\right|\\
    &= d(x, y) + d(y, z)
\end{align*}

Thus, $d$ is a metric on $X$. 
\end{document}


