%        File: notes.tex
%     Created: Tue Mar 24 01:00 PM 2015 C
% Last Change: Tue Mar 24 01:00 PM 2015 C
%
\documentclass[a4paper, twocolumn]{report}

\usepackage[]{amsthm} 
\usepackage[]{amsmath} 
\usepackage[]{amssymb} 
\usepackage[]{hyperref} 
\usepackage[]{enumerate} 
\usepackage[]{todo} 

\newcounter{dummy} \numberwithin{dummy}{section}
\newcounter{exercise} \numberwithin{exercise}{section}
\newcounter{axmcntr} \numberwithin{axmcntr}{chapter}
\newtheorem{axm}[axmcntr]{Axiom}
\newtheorem{prp}[dummy]{Proposition}
\newtheorem{asm}[axmcntr]{Assumption}
\newtheorem{lma}[dummy]{Lemma}
\newtheorem{crl}[dummy]{Corollary}
\newtheorem*{crl*}{Corollary}
\theoremstyle{definition}
\newtheorem{defn}[dummy]{Definition}
\newtheorem{exc}[exercise]{Exercise}
\makeatletter
\def\@endtheorem{\qed\endtrivlist\@endpefalse }
\makeatother
\newtheoremstyle{solution}{}{}{}{}{\scshape}{:}{ }{}
\theoremstyle{solution}
\newtheorem*{sltn}{Solution}

\newcommand{\dplus}{{+}{+}} % Double unary plus sign.
\newcommand{\union}{\cup}
\newcommand{\intrsct}{\cap}


\title{
  \textsc{University of Oslo} \\
  \textsc{Real Analysis - MAT2400} \\
  \textsc{Spring 2015}\\
}
\author{Ivar Stangeby}
\date{\today}

\begin{document}

\twocolumn[
  \begin{@twocolumnfalse}
    \maketitle
    \begin{abstract}
      This document is going to be a way for me as a student of MAT2400 at the
      University of Oslo to gather my thoughts around the course Real Analysis.  I've
      struggled with my intuition for this subject, and this is a last ditch effort
      to build it all up from scratch.

      For this task, I've decided to use the text books written by Terence Tao,
      namely Analysis I and II. There is absolutely nothing wrong with the text book
      offered at my university, it is brilliantly written, but I have been exposed to
      the writings of Terence Tao before, and therefore I wish to give his books a
      try.

      The first part of this document related to the book Analysis I, is going to be
      mostly involved in building a solid foundation for the concepts discussed in
      Analysis II. The material included from Analysis I is very similar to the
      curriculum of the subject MAT1140 at the University of Oslo.

      The structure of this document is going to be me writing down the results
      encountered throughout the text books along the proofs I find extra intriguing.
      I'm going to attempt to prove the theorems myself, and if I find it reasonable
      I'm going to write down my own proof. Included will also be my attempted
      solutions to selected exercises. 

      This document is mainly for my own good and well being, but if anyone can find
      any use from them, then that is great.
    \end{abstract}
  \end{@twocolumnfalse}
]

\tableofcontents

\chapter{Metric spaces}

\end{document}
