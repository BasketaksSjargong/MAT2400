\documentclass{article}

\usepackage{fontenc} 
\usepackage{fontspec} 

\usepackage{amsthm} 
\usepackage{amsmath} 
\usepackage{amssymb} 


\usepackage{hyperref} 
\usepackage{cleveref} 

\title{Mandatory Assignment --- MAT2400}
\author{Ivar Haugaløkken Stangeby}

\begin{document}
    \maketitle

    \section*{Problem 1}
    
    In this problem we consider the series given by
    \begin{equation}
        \label{eq:series}
        \sum^{\infty}_{n=1} \frac{1}{1 + n^2x}.
    \end{equation}
    First and foremost, we are interested in under what values of $x$ this
    series converge.  We see that by setting $x = 0$, then the series read $1 +
    1 + 1 + \ldots$ which clearly sums to infinity. So we can conclude that for
    $x = 0$ the series diverges. For $x > 0$ we now have a contribution from
    the $n$'s again.  We know that $1 /(1 + n^2x)$ is certainly smaller than $1
    / (n^2x)$ for all $x > 0$.  Hence we have the following inequality:
    \begin{equation}
        \notag
        \sum^{\infty}_{n=1} \frac{1}{1 + n^2x} 
        < \sum^{\infty}_{n = 1} \frac{1}{n^2x} 
        = \frac{1}{x} \sum^{\infty}_{n=1} \frac{1}{n^2} = \frac{M}{x}.
    \end{equation}
    Hence no matter what $x$ we chose, the limit is always smaller than $M /
    x$.\footnote{$M$ was shown by Euler to be equal to $\pi^2 / 6$.} 
    Hence the series converges for any $x > 0$.

    We are now interested in whether the convergence is uniform or not on the
    interval $[a, \infty)$. Weierstrass' $M$-test immediately tells us that the
    convergence is uniform for the interval under the condition that $a
    \geq 1$. The problem lies in the area between zero and one.

\end{document}

