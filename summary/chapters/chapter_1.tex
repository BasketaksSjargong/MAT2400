\chapter{Preliminaries: Proofs, Sets, and Functions}

\section{Proofs}

\subsubsection{Implication}
Many mathematical statements are on the form \textit{If A, then B}. These are
denoted \( A \Rightarrow B \). If \( A \Rightarrow B \) then it is not neccessarily
so that \( B \Rightarrow A \). That is, these two mean different things.
Note that \( A \Rightarrow B \) is logically equivalent to \( \sim B \Rightarrow \sim A \).

\subsubsection{Equivalence}
If it is true that \( A \Rightarrow B \) and \( B \Rightarrow A \). Then we call these
statements equivalent. This is denoted \( A \Longleftrightarrow B \).
When proving that two statements are equivalent, it is often easier to prove that
they both imply eachother. That is, by proving \( A \Rightarrow B \) and \( B \Rightarrow A \), you have proven that \( A \Longleftrightarrow B \).

\subsubsection{Methods of proof}
There are several ways of proving mathematical hypotheses.
\begin{enumerate}
    \item 
Instead of proving \( A \Rightarrow B \), prove \( \sim B \Rightarrow \sim A \).
This is called a \textit{contrapositive proof}.
    \item
Another common method of proof is \textit{proof by contradiction}. Assume the opposite of what you want to prove, and by showing that this leads to a contradiction, our assumption must be false, and hence the original hypothesis is true.
    \item
When dealing with natural numbers, the go-to method of proof is \textit{proof by induction}. Show that a statement holds for a given number, and then show that if it holds for an arbitrary number \(n\), it must also hold for the successor \( n+1 \).
\end{enumerate}
This list is of course non-exhaustive, but these are some of the more common methods one should consider when faced with a problem to solve. 

\section{Sets and boolean operations}

A set is a collection of mathematical objects. It may be finite, it may be infinite.
\( A \) is a subset of \( B \) if all the elements in \( A \) are also in \( B \). This is denoted \( A \subseteq B \).
Two sets are equal if they contain exactly the same elements. This is denoted \( A = B \), where \( A \) and \( B \) are sets. If \( A \) is a subset of \( B \) and \( B \) is a subset of \( A \), the sets are equal.

The set that contains no elements is called the empty set and is denoted \( \emptyset \).
\section{Families of sets}
\section{Functions}
\section{Relations and partitions}
\section{Countability}
