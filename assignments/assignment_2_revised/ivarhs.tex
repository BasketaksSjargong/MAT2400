%        File: ivarhs.tex
%     Created: Mon Apr 27 02:00 PM 2015 C
% Last Change: Mon Apr 27 02:00 PM 2015 C
%
\documentclass[a4paper]{article}

\usepackage[]{amsmath} 
\usepackage[]{amsthm} 
\usepackage[]{amssymb} 
\usepackage[]{fourier} 
\usepackage[T1]{fontenc} 
\newcommand{\dist}{\text{dist}}

\title{
  Mandatory assignment 2\\
  MAT2400
}
\author{Ivar Stangeby}
\begin{document}
\maketitle
\section*{Problem 1}
\paragraph{a)}
  We want to show that $h(A, B) = 0$ implies that $A = B$. We assume that $A$
  and $B$ are two non-empty closed subsets of $X$. In other words, $A, B \in
  P(X)$.  If we can show the contrapositive, then the original implication must
  hold.  We therefore assume that $A \neq B$ and we want to show that then
  $h(A, B) \neq 0$.
  
  Let $x\in X$ be such that $x \in A$ and $x \neq B$. By the definition of
  $\dist(x, A)$ and $\dist(x, B)$ we see that since $x \in A$, $\dist(x, A) =
  0$ and since $x$ is not in $B$ we have $\dist(x, B) > 0$ by the definition of
  $d$. Since we take the supremum of all of these values we must have that
  \begin{equation}
    \notag
    \sup_{x \in X}|\dist(x, A) - \dist(x, B)| \geq \dist(x, B) > 0.
  \end{equation}

  Hence, $h(A, B) = 0$ implies that $A = B$.

\paragraph{b)}
We now want to show that $h$ is a metric on $P(X)$. Positivity and symmetry
follows directly from the absolute values in the expression for $h(A, B)$. 
We therefore need to show that $h$ obeys the triangle inequality. 

\begin{align*}
  h(A, B) &= \sup_{x \in X}|\dist(x, A) - \dist(x, B)| \\
  &= \sup_{x \in X}|\dist(x, A) - \dist(x, C) + \dist(x, C) - \dist(x, B)|\\
  &\leq \sup_{x \in X}\left(| \dist(x, A) - \dist(x, C)| + |\dist(x, C) - \dist(x, B)|\right)\\
  &\leq \sup_{x \in X}| \dist(x, A) - \dist(x, C)| + \sup_{x \in X}|\dist(x, C) - \dist(x, B)|\\
  &= h(A, C) + h(C, B)
\end{align*}
Hence $h$ is a metric on $P(X)$.

\paragraph{c)}
We want to show the two inequalities
\begin{equation}
  h(A, B) \geq \hat{h}(A, B), 
\end{equation}
\begin{equation}
  \hat{h}(A, B) \geq h(A, B).
\end{equation}

\section*{Problem 2}
\paragraph{a)}
\subsubsection*{Convergence}
We want to show that the series
\begin{equation}
  \notag
  \Upsilon = \sum_{n = - \infty}^{\infty}r^{|n|}e^{inx}
\end{equation}
converges uniformly for all $x \in \mathbb{R}$ and that its sum equals $P_r(x)$. 
We observe that we can split $\Upsilon$ into three subseries:\footnote{Note the change of summation index and the removal of absolute values}
\begin{equation}
  \notag
  \Upsilon = 1 + \sum_{n=1}^{\infty}r^{n}e^{inx} + \sum_{n=1}^{\infty}r^{n}e^{-inx}.
\end{equation}
Applying Eulers formula, we can simplify this to the series\footnote{The imaginary terms cancel out}
\begin{equation}
  \notag
  \Upsilon = 1 + \sum_{n=1}^{\infty}r^n\cos(nx) + \sum_{n=1}^{\infty}r^n\cos(nx) = 1 + 2 \sum_{n=1}^{\infty}r^n\cos(nx).
\end{equation}
We know that since we have $0 < r < 1$ and $\cos(nx) \leq 1$ that
\begin{equation}
  \notag
  \Upsilon \leq 1 + 2\sum_{n=1}^{\infty}r^n, 
\end{equation}
but since this series converges, so must $\Upsilon$. 

\subsubsection*{Sum}

We want to show that the sum of $\Upsilon$ is equal to
\begin{equation}
  \notag
  P_r(x) = \frac{1 - r^2}{1 - 2r\cos(x) + r^2}.
\end{equation}
We rewrite $\Upsilon$ to the following form, in order for us to be able to apply the formula for
finite geometric series:\footnote{Notice again the change of summation index. That is where the $-1$ originates from.}
\begin{equation}
  \notag
  \Upsilon = \lim\limits_{N \rightarrow \infty}\left(\sum_{n=0}^{N}\left( e^{ix}r \right)^n + \sum_{n=0}^{N}\left(e^{-ix}r\right)^{n} - 1\right).
\end{equation}
We then get that the sum of $\Upsilon$ is equal to
\begin{equation}
  \notag
  \Upsilon = \lim\limits_{N\rightarrow\infty} \left(\frac{1 - (e^{ix}r)^{N+1}}{1 - e^{ix}r} + \frac{1-\left( e^{-ix}r \right)^{N+1}}{1 - e^{-ix}r} - 1\right).
\end{equation}
Applying Eulers formula again which yields\footnote{Some algebra required}
\begin{equation}
  \notag
  \Upsilon = \frac{1 - r^{2}}{1 - 2r\cos(x) + r^{2}} = P_r(x).
\end{equation}

\paragraph{b)}
We want to show that $P_r(x)$ is positive or zero for all $x$. 
We have
\begin{align*}
  P_r(x) &= \frac{1 - r^{2}}{1 - r2\cos(x) + r^{2}} \\
  &\geq \frac{1 - r^{2}}{1 - 2r + r^{2}}\\
  & = \frac{(1 - r)(1 + r)}{(1-r)^{2}} = \frac{1 + r}{1 - r} > 1.
\end{align*}
Therefore, we can conclude that no matter what $x$ is, $P(x)$ is greater or
equal to zero.

\paragraph{c)}
We want to show that $P_r(x) \rightarrow 0$ as $r\uparrow 1$ on the interval $X = \left[ -\pi, -\delta \right]\cup\left[ \delta, \pi \right]$.

We first state the definition of uniform convergence.\footnote{This is a bit informal, but used just as a reminder}

\begin{quote}
  A sequence $\left\{ f_n \right\}$ of functions converges uniformly to a function $f$
  if and only if for all $\varepsilon > 0$ there exists an $N > 0$ such that $|f_n(x) - f(x)| < \varepsilon$ for all $x \in X$ and for all $n \geq N$.
\end{quote}

In our case, we need to rewrite $P_r(x)$ to a form which we can express in terms of a natural number $n$.
We want to create the sequence $\left\{ r_n \right\}$ of rational numbers defined as
\begin{equation}
  \notag
  r_n = \frac{n - 1}{n}.
\end{equation}
This series converges to $1$ as $n \rightarrow \infty$. We now want to show that for all $\varepsilon>0$ there exists an $N > 0$ such that what was stated above holds. That is
\begin{equation}
  \notag
  |P_{r_n}(x) - 0| < \varepsilon, 
\end{equation}
for all $x \in X$ and for all $n > N$.
Writing it out and taking the limit we see that
\begin{align*}
  \lim\limits_{n\rightarrow\infty}\frac{1 - \left(\frac{n - 1}{n}\right)^2}{1 - \left( \frac{n-1}{n} \right)\cos(x) + \left( \frac{n-1}{n} \right)^{2}} = \frac{0}{2 - 2\cos(x)} = 0 < \varepsilon.
\end{align*}
We have $\cos(x) \neq 1$ for all $x \in X$ since $0 \notin X$, therefore our proof is done.

\paragraph{d)}
We want to show that
\begin{equation}
  \notag
  \int\limits_{-\pi}^{\pi}P_r(x)\, dx = 2\pi.
\end{equation}
Recall that
\begin{equation}
  \notag
  P_r(x) = 1 + 2\sum_{n=1}^{\infty}r^n\cos(nx).
\end{equation}
We can therefore write the integral as follows
\begin{equation}
  \notag
  \int\limits_{-\pi}^{\pi} \left(1 + 2\sum_{n=1}^{\infty}r^n\cos(nx)\right) \, dx = \int_{-\pi}^{\pi} 1 dx + 2\int_{-\pi}^{\pi}\left(\sum_{n=1}^{\infty}\left(r^n\cos(nx)\right)\right)dx
\end{equation}

We know that since $P_r(x)$ converges uniformly on all of $\mathbb{R}$ the sum and integrals are interchangeable.\footnote{Corollary 4.2.3}
Therefore we can rewrite this expression as
\begin{equation}
  \notag
  \int_{-\pi}^{\pi} 1 \, dx + 2\sum_{n=1}^{\infty}\left( \int_{-\pi}^{\pi}r^n\cos(nx) \, dx \right).
\end{equation}
The integral contained in the sum evaluates to zero, therefore we are left with
\begin{equation}
  \notag
  \int_{-\pi}^{\pi}P_r(x) \, dx = \int_{-\pi}^{\pi}1 \, dx = 2\pi, 
\end{equation}
and we are done.

\paragraph{e)}
  
We want to show that 
\begin{equation}
  \notag
  \Omega = \lim\limits_{r\uparrow 1} \frac{1}{2\pi}\int_{-\pi}^{\pi}f(x-y)P_r(y) \, dy = f(x).
\end{equation}

Taking the limit as $r \uparrow 1$, our expression for $P_r(y)$ turns into something we recognise
as the Dirichlet kernel, $D_N(y) = \sum_{n=-N}^{N}e^{iny}$.  \footnote{I am here following the derivation at page 138 and 139 in the compendium}
\begin{equation}
  \notag
  \lim\limits_{r\uparrow 1} P_r(y) = \lim\limits_{N\rightarrow\infty}\sum_{n = -N}^{N}e^{iny} = \lim_{N\rightarrow \infty}D_N(y).
\end{equation}
We can now look at the partial sums of $\Omega$:
\begin{equation}
  \notag
  s_N(x) = \frac{1}{2\pi}\int_{-\pi}^{\pi}f(x - y)D_N(y)\, dy.
\end{equation}
If we can show that these partial sums converge to $f(x)$ we're done.  Using
Dirichlet's Theorem we know that if $f$ has a finite number of minima and
maxima, then the Fourier series of $f$ converges pointwise to $f$. Since $f$ is
$2\pi$ periodic, it has a finite number of min and max. Therefore, we have
\begin{equation}
  \notag
  \Omega = \lim\limits_{N \rightarrow \infty} s_N(x) = \lim\limits_{r\uparrow 1}\frac{1}{2\pi}\int_{-\pi}^{\pi}f(x - y)P_r(y) \, dy = f(x).
\end{equation}
This holds because we actually looking at partial sums $s_N$ of the Fourier series for $f$, 
\begin{equation}
  \notag
  s_N(x) = \sum_{n=-N}^{N}\langle f, e_n \rangle e_n(x).
\end{equation}
\end{document}


