%        File: solution.tex
%     Created: Tue Feb 24 01:00 PM 2015 C
% Last Change: Tue Feb 24 01:00 PM 2015 C
%
\documentclass[a4paper]{article}
\usepackage{amsmath}
\usepackage{amsthm}
\usepackage{amsfonts}
\newtheorem{prb}{Problem}

\title{
  \huge MAT2400\\
  \centering{Assignment 1}
  \author{Ivar Stangeby}
  \date{\today}
}
\begin{document}
\maketitle
\begin{prb}
  Show that a strictly increasing function $f: \mathbb{N} \rightarrow
  \mathbb{N}$ must satisfy $f(n) \geq n, \forall n \in \mathbb{N}$.
\end{prb}

\paragraph{Solution: }
Assume that $f: \mathbb{N} \rightarrow \mathbb{N}$ is a strictly increasing
function.  By the definition of a strictly increasing function we know that
\begin{equation}
  \notag
  f(n + 1) > f(n), \forall n \in \mathbb{N}.
\end{equation}

We can now, since we are working with the natural numbers, easily show this
inductively.  Let us first show the base case.
\begin{align*}
  \label{eq:}
  f(1) \geq 1  
\end{align*}

This is intuitively true, because $1$ is defined as the least element of the
set of natural numbers.  Assuming that we have verified this as true for all
$n$ up to and including some number $k$.  We know want to show that it then
follows that it must be true for $k + 1$. 
By assumption:
\begin{equation}
  \notag
  f(k) \geq k
\end{equation}
Using the standard metric in $\mathbb{R}$ we can see that for any two pairs of successive
integer numbers, 
\begin{align*}
  \inf\left\{ d(k, k+1) \mid k \in \mathbb{N} \right \} = 1 \\
  \intertext{ where,}
  d(x, y) = |x - y|
\end{align*}

That is, the smallest distance possible with two different numbers is 1.  It
then follows that
\begin{align*}
  \label{eq:}
  f(k+1) > f(k) + 1 \geq k+1 \\
  f(k+1) \geq k+1
\end{align*}
as we wanted to show.  Thus, by the induction principle, a strictly increasing
function from $\mathbb{N}$ to $\mathbb{N}$, must neccessary satisfy $f(n) \geq
n, \forall n \in \mathbb{N}$.

\begin{prb}
Let $\left( X, d) \right)$ be a complete metric space. Let $B(x, r)$ denote the open ball
centered at $x \in X$ with radius $r$, i.e., 
\begin{equation}
  \notag
  B(x, r) = \left\{ y \in X \mid d(x, y) < r \right\},
\end{equation}
and $\overline{B}\left( x, r \right)$ the closed ball of radius $r$, i.e.,
\begin{equation}
  \notag
  \overline{B}\left( x, r \right) = \left\{ y \in X \mid d(x, y) \leq r \right\}.
\end{equation}
For any set $C \in X$, let $\overline{C}$ denote its closure.
Is it true that for any complete metric space $X$, 
\begin{equation}
  \label{eq:condition}
  \overline{B\left( x, r \right)} = \overline{B}\left( x, r \right)?
\end{equation}
\end{prb}

\paragraph{Solution: }

Consider the discrete metric,
\begin{equation}
  \notag
  d(x, y) = \begin{cases}
    0 & \text{if } x = y \\
    1 & \text{if } x \neq y
  \end{cases}
\end{equation}

We can show that \eqref{eq:condition} does not neccessarily hold under the
discrete metric.  Lets assume we take the radius $r$ to be $1$.  The open ball
$B(x, r)$ is then any two points with less than a distance $r$ between.  Thus
the open ball only contains the point $x$. In that case, taking the closure of
this open ball changes nothing, and we're left with just the point $x$.
However, the closed ball $\overline{B}\left( x, r \right)$ has to be the
entirety of our space $X$, since the distance between two points are allowed to
be $1$.  Thus, if we let our metric space be $(X, d)$ with $X = \mathbb{R}$ and
$d$ the discrete metric \eqref{eq:condition} does not hold. We then have a
complete metric space $(R, d)$. We then have a complete metric space $(R, d)$.

\begin{prb}
Let $\ell$ be the set of sequences of real numbers where only a finite number of terms
are different from zero,
\begin{equation}
  \notag
  \ell = \left\{ \left\{ x_n \right\}_{n=1}^{\infty} \mid x_i = 0 \text{ for all but a finite number of $i$'s} \right\}.
\end{equation}
For $x = \left\{ x_n \right\}$ and $y = \left\{ y_n \right\}$ in $\ell$, define
\begin{equation}
  \notag 
  d(x, y) = \sup_{n\in\mathbb{N}} |x_n - y_n|.
\end{equation}
\end{prb}

\paragraph{a)}

To show that $d$ is a metric on $\ell$ we must show the three properties of a metric function.
\begin{enumerate}
  \item Positivity:
    Since the metric is defined as the biggest difference between to
    corresponding elemnts from $\left\{ x_n \right\}$ and $\left\{ y_n
    \right\}$, the metric must neccesarily satisfy the property of positivity
    since there does exists a finite number of non-zero elements in each sequence.
    Thus, $d(x, y) \geq 0$ with equality only if $x = y$. 
  \item Symmetry:
    
    \begin{equation}
      \notag
      d(x, y) = \sup_{n\in\mathbb{N}} |x_n - y_n| = \sup_{n\in\mathbb{N}}|y_n - x_n| = d(y, x).
    \end{equation}
    Thus the metric is symmetric.

  \item Triangle Inequality:
    Want to show that given three sequences $x, y, z$, the metric satisfies
    \begin{equation}
      \notag
      d(x, z) \leq d(x, y) + d(y, z).
    \end{equation}
    The trival case, if $x = y = z$, we know that $d(x, y) = d(y, z) = d(x, z) = 0$.
\end{enumerate}
\end{document}


