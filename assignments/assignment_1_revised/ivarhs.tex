%        File: ivarhs.tex
%     Created: Mon Mar 02 07:00 PM 2015 C
% Last Change: Mon Mar 02 07:00 PM 2015 C
%
\documentclass[a4paper]{article}

% Mathematical typsetting
\usepackage{amsmath}
\usepackage{amsthm}
\usepackage{amssymb}

% Language and encoding
\usepackage[utf8]{inputenc}

% Using theoremstyles for minisectioning
% Avoid clunky section commands for solutions of problems
\newtheorem{prb}{Problem}
\theoremstyle{definition}
\newtheorem{sol}{Solution}

% Title
\title{
  \huge{\textsc{Mandatory Assignment  MAT2400}} \\
  \textsc{University of Oslo}
}
\author{Ivar Haugaløkken Stangeby}
\date{\today}

\begin{document}
\maketitle
\begin{prb}
  Assume $f : \mathbb{N} \rightarrow \mathbb{N}$ is a strictly increasing function.
  Show that $f$ satisfies $f(n) \geq n$ for all $n$.
\end{prb}

\begin{sol}
This obviously holds for $n = 1$, since $1$ is the least element of
$\mathbb{N}$.  Assume now that we have shown $f(n) \geq n$ for all $n$ up to
some $n = k$.  Since $f(k+1) > f(k)$, by the assumption that $f$ is strictly
increasing we can easily see that $f(k+1) - f(k) > 1$. This is because for two
elements to be unequal in $\mathbb{N}$, they must differ by at least $1$.

\begin{equation}
  \notag
  f(k+1) > f(k) \geq k \Longrightarrow f(k+1) \geq k + 1.
\end{equation}
Thus it must neccessarily hold for $n = k+1$ as well.
By induction we have now shown that $f$ satisfies $f(n+1) \geq n$ for all $n$.
\end{sol}

\begin{prb}
  Assume $\left( X, d \right)$ a complete metric space. Prove or disprove that the closure of
  an open ball $\overline{ B\left( x, r \right)}$ is equal to the closed ball
  $\overline{ B }\left( x, r \right)$ for all complete metric spaces.
\end{prb}

\begin{sol}
  We can disprove the above by looking at a complete space under the discrete metric.
  Let us first introduce the metric $d : X \times X \rightarrow \left\{ 0, 1 \right\}$, defined
  as
  \begin{equation}
    \notag
    d(x, y) = \begin{cases}
                0, & x = y \\
                1, & x \neq y
              \end{cases}
  \end{equation}

  To construct a counter example, assume that we look at open at closed balls
  around a point $x$, with a radius $r = 1$. By observation we see that the
  closure of the open ball around $x$ is just $x$ itself, 
  \begin{equation}
    \notag
    \overline{ B\left( x, r \right)} = x
  \end{equation}
  where as the closed ball around $x$ is the whole of $X$. 
  \begin{equation}
    \notag
    \overline{B}\left( x, r \right) = X.
  \end{equation}

  We have now, using the discrete metric $d$ and a radius $r = 1$ constructed a
  metric space where the statement initially given does not hold.
\end{sol}

\begin{prb}
  Define $\ell$ to be the set of all sequences of real numbers where only a
  finite number of elements are non-zero. Furthermore, for $x = \left\{ x_n \right\}$ and
  $y = \left\{ y_n \right\}$ in $\ell$, define
  \begin{equation}
    \notag
    d(x, y) = \sup\limits_{n\in\mathbb{N}} \left| x_n - y_n\right|.
  \end{equation}
\end{prb}
\paragraph{a)}

To show that $d$ defines a metric on $\ell$ we have to show the three
properties a metric should satisfy. These are \textit{positivity, symmetry} and
\textit{the triangle inequality.} The first two, positivity and symmetry follow
directly from the definition of absolute value.  The triangle inequality is a
bit trickier to show. We want $d$ to satisfy the following relation:

\begin{equation}
  \notag
  d(x, z) \leq d(x, y) + d(y, z).
\end{equation}
Written out, this becomes:

\begin{equation}
  \notag
  \sup\limits_{n\in\mathbb{N}} \left| x_n - z_n\right| \leq \sup\limits_{n\in\mathbb{N}} \left| x_n - y_n\right| +  \sup\limits_{n\in\mathbb{N}} \left| y_n - z_n\right|.
\end{equation}
We know that for some $i, j, k \in \mathbb{N}$:

\begin{align*}
  \label{eq:}
  &d(x, z) = \left| x_i - z_i \right| \\
  &d(x, y) = \left| x_j - y_j \right| \\
  &d(y, z) = \left| y_k - z_k \right| \\
\end{align*}
Intuitively, these absolute values should obey the triangle inequality for real
numbers, but I can't say I've managed to completely convince myself. I have
therefore not been able to construct a full fledged proof for $d$ being a
metric on $\ell$.

\paragraph{b)}
Let $u_k \in \ell$ be defined as
\begin{equation}
  \notag
  u_k = \left\{ 1, \frac{1}{2}, \frac{1}{3}, \cdots, \frac{1}{k}, 0, 0, 0, \cdots \right\}.
\end{equation}
To show that $\left\{ u_k \right\}_{ k = 1 }^{\infty}$ is Cauchy we want to
show that for all $\varepsilon > 0$ there exists an $N \in \mathbb{N}$ such
that for all $m, n > N$, it is implied that $d\left( x_n, x_m \right) <
\varepsilon$.

Given two sequences $u_n$ and $u_m$, the distance between them is given as
following (assuming $m > n$).

\begin{align*}
  \notag
  d(u_n, u_m) &= \sup_{i\in\mathbb{N}}\left| u_{n_{i}} - u_{m_{i}} \right| \\ 
              &= | u_{n_{n+1}} - u_{m_{n+1}} | \\
              &= | 0 - \frac{1}{n+1} |\\
              &= \frac{1}{n+1}
\end{align*}
Since a number on this form can me made as arbitrarily close to zero by chosing
$n$ arbitrarily large, we have shown that no matter what $\varepsilon > 0$ we
are given, we can always find an $N$ such that $m, n > N \Longrightarrow d(u_m,
u_n) < \varepsilon$. Because of this, $\left\{ u_k \right\}_{n=1}^{\infty}$ is
Cauchy.

An alternative, but equivalent, way of doing it would be to generate the sequence

\paragraph{c) - This one is not yet complete}

One way of check whether $\left\{ u_k \right\}$ is convergent would be to look
at the compactness, and therefore indirectly, the completeness of $\ell$. If
one can show that $\ell$ is compact, then by definition of \textit{complete},
all Cauchy sequences in $\ell$ must converge.  Since $\left\{ u_k \right\}$ is
Cauchy, it would follow that also $\left\{ u_k \right\}$ converges.

Another, perhaps simpler way is to show that we can find a subsequence of
$\left\{ u_k \right\}$, and show that it converges to a point $a$ in $\ell$. If
this is the case, then the original sequence $\left\{ u_k \right\}$ must also
converge to $a$. Doing it this way we don't have to show anything about the
completeness or compactness of $\ell$.  In order for $\left\{ u_k \right\}$ to
converge to $a$, the real sequence $\left\{ d(u_k, a)\right\}$ must converge to
$0$.\\ Rephrased:

\begin{equation}
  \notag 
  \left\{ u_k \right\} \longrightarrow a \Longleftrightarrow \left\{ d(u_k, a) \right\} \longrightarrow 0
\end{equation}

If we let $a$ be equal to the null-sequence in $\ell$. That is
\begin{equation}
  \notag
  a = \left\{ 0, 0, 0, \ldots \right\}
\end{equation}

I feel there is some clever result here that I haven't yet discovered. 
\paragraph{d)}
Define $c_0$ to be the space of sequences of real numbers whose limit are 0.
\begin{equation}
  \notag
  c_{0} = \left\{ \left\{ x_n \right\}_{n=1}^{\infty} \mid \lim\limits_{n\rightarrow\infty} x_n = 0 \right\}.
\end{equation}
To show that $\left( c_0, d \right)$ is a complete metric space, we must show
that all Cauchy sequences in $\left( c_0, d \right)$ converge.

We know by definition of $c_0$ that all its elements are sequences of real
numbers that converge to $0$.  If a sequence of real numbers converges to zero,
it is also Cauchy. Therefore, it follows that $c_0$ is a complete metric space.

\paragraph{e)}

By the definition of $c_0$ it is clear that $\ell \subset c_0$.  $\ell$ is
dense in $c_0$ if and only if for every $x \in c_0$ and for all $\varepsilon >
0$ there exists a $y \in \ell$ such that $d(x, y) < \varepsilon$.  Intuitively
this seems reasonable, since we know a sequence $x$ converges to $0$, and $y$
contains an infinite number of $0$'s.
\begin{align*}
x &= \left\{ x_1, x_2, x_3, \ldots \right\} \longrightarrow 0\\
y &= \left\{ y_1, y_2, y_3, \ldots, y_n, 0, 0, 0, \ldots \right\} \longrightarrow 0
\end{align*}
Thus, we can always chose the sequences we need to satisfy the definition of
\textit{dense}.  Therefore, $\ell$ is dense in $c_0$.

The limit of the sequence $u_k$, containing an infinite amount of zeroes, is
zero.  That is $u_k \longrightarrow 0$. This means that $u_k \in c_0$.

\begin{prb}
  Let $X = \left( 0, \infty \right) \subset \mathbb{R}$ and let $d : X \times X \longrightarrow \mathbb{R}$ be defined as
  \begin{equation}
    \notag
    d(x, y) = | \ln(x) - \ln(y)|
  \end{equation}
\end{prb}

\paragraph{a)}
We want to show that $d$ is a metric, and that $\left( X, d \right)$ is complete.
First, we need to show the three properties of a metric function. The first two, positivity and symmetry
are trivial.
For the triangle inequality, we observe the following:

\begin{align*}
  d(x, z) &= | \ln(x) - \ln(z)| = | \ln(x) + \ln(y) - \ln(y) - \ln(z) | \\
          &\leq |\ln x - \ln(y) | + |\ln(y) - \ln(z) | \\
          &= d(x, y) + d(y, z)
\end{align*}
Thus $d$ is a metric on $X$. To show that $\left( X, d \right)$ is complete, we
must show that all Cauchy sequences converge.

\end{document}




