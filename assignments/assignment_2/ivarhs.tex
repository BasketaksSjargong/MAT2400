%        File: ivarhs.tex
%     Created: Sat Apr 04 04:00 PM 2015 C
% Last Change: Sat Apr 04 04:00 PM 2015 C
%
\documentclass[a4paper]{article}
\usepackage[]{amsmath} 
\usepackage[]{amssymb} 
\usepackage[]{amsthm} 
\usepackage[]{enumitem} 

\usepackage[utf8]{inputenc} 

\theoremstyle{definition}
\newtheorem{prb}{Problem}
\newtheorem{sltn}{Solution}

\newcommand{\dist}{\text{dist}}
\newcommand{\rimply}{\Longrightarrow}
\newcommand{\limply}{\Longleftarrow}
\newcommand{\equivalence}{\Longleftrightarrow}

\setlist[enumerate, 1]{label=\bfseries\alph*), leftmargin=*}

\title{
  \textsc{University of Oslo}\\
  \textsc{Real Analysis - MAT2400}\\
  \textsc{Assignment 2}\\
}
\author{Ivar Haugaløkken Stangeby}
\date{\today}

\begin{document}

\maketitle

\begin{prb}
  Let $\left( X, d \right)$ be a bounded metric space, and let $P\left( X \right)$ denote the
  collection of non-empty closed subsets of $X$. For $A$ and $B$ in $P\left( X \right)$, let
  \begin{equation}
    \notag
    h\left( A, B \right) = \sup_{x\in X} \left| \dist(x, A) - \dist(x, B) \right|,
  \end{equation}
  where $\dist(x, C)$ is given by
  \begin{equation}
    \notag
    \dist\left( x, C \right) = \inf_{c\in C}d(x, c).
  \end{equation}
  The function $h$ is called the \textit{Hausdorff metric}.
  \begin{enumerate}
    \item Show that if $h(A, B) = 0$ then $A = B$. Here $A$ and $B$ are two non-empty closed subsets of $X$.
    \item Show that $h$ is a metric on $P\left( X \right)$.
    \item For $A$ and $B$ in $P\left( X \right)$, let $\hat{h}$ be defined as
      \begin{equation}
        \notag
        \hat{h}\left( A, B \right) = \max \left\{ \sup_{a \in A}\dist\left( a, B \right), \sup_{b\in B}\dist\left( b, A\right) \right\}.
      \end{equation}
  \end{enumerate}
  Show that
  \begin{equation}
    \notag
    \hat{h}\left( A,B \right) = h\left( A, B \right) \text{ for all } A, B \text{ in } P\left( X \right).
  \end{equation}
  (\textbf{Hint:} Show the two inequalities $h\left( A, B \right) \geq \hat{h}\left( A, B \right)$ and $\hat{h}\left( A, B \right) = h\left( A, B \right)$.) 
\end{prb}

\begin{sltn}
  Before I start, I want to jot down the properties of the various mathematical
  objects presented.  We are given a metric space $\left( X, d \right)$ and it
  is said to be bounded. What this means, is that there exists some number $r$
  such that $d(x, y) \leq r$ for all $x, y \in X$.  $P\left( X \right)$ is a
  collection of non-empty closed subsets of $X$.  In other words, the elements
  of $P\left( X \right)$ are sets that contain their own boundary.

  We want to show $h(A, B) = 0 \rimply A = B$. We want to show $A \neq B
  \rimply h(A, B) \neq 0$.
  From the left side of our implication we have
  \begin{equation}
    \notag
    \exists y \in X : y \notin A \lor y \notin B
  \end{equation}

  Let us now look at $|\dist(y, A) - \dist(y, B)|$.
  \begin{align*}
    |\dist(y, A) - \dist(y, B)| = |\inf_{a\in A}d(y, a) - \inf_{b\in B}d(y, b)|
    =  |d(y, a') - d(y, b')|
  \end{align*}, for some $a' \in A$ and some $b' \in B$. 

  Since $y \notin A$ and $A$ is closed we can construct a ball $B(y,
  \varepsilon_2)$ such that $B(y, \varepsilon) \cap A = \emptyset$.  Therefore,
  $d(y, a) > \varepsilon_1$. Similarly we can construct a ball around $y$ whose
  intersection with $B$ is empty, and therefore $\dist(y, A) > 0$ and $\dist(y,
  B) > 0$.  
\end{sltn}

\begin{prb}
  Let $0 < r < 1$ and consider the series
  \begin{equation}
    \notag
    \sum_{n=-\infty}^{\infty}r^{|n|}e^{inx}.
  \end{equation}
  Show that the series converges uniformly for all $x \in \mathbb{R}$, and that
  its sum equals
  \begin{equation}
    \notag
    P_{r}(x) = \frac{1-r^{2}}{1 - 2r \cos(x)+r^{2}}.
  \end{equation}
\end{prb}
\end{document}

