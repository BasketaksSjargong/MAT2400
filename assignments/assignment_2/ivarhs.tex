%        File: ivarhs.tex
%     Created: Sat Apr 04 04:00 PM 2015 C
% Last Change: Sat Apr 04 04:00 PM 2015 C
%
\documentclass[a4paper]{article}
\usepackage[]{amsmath} 
\usepackage[]{amssymb} 
\usepackage[]{amsthm} 
\usepackage[]{enumitem} 

\usepackage[utf8]{inputenc} 

\theoremstyle{definition}
\newtheorem{prb}{Problem}
\newtheorem{sltn}{Solution}

\newcommand{\dist}{\text{dist}}
\newcommand{\rimply}{\Longrightarrow}
\newcommand{\limply}{\Longleftarrow}
\newcommand{\equivalence}{\Longleftrightarrow}

\setlist[enumerate, 1]{label=\bfseries\alph*), leftmargin=*}

\title{
  \textsc{University of Oslo}\\
  \textsc{Real Analysis - MAT2400}\\
  \textsc{Assignment 2}\\
}
\author{Ivar Haugaløkken Stangeby}
\date{\today}

\begin{document}

\maketitle

\begin{prb}
  Let $\left( X, d \right)$ be a bounded metric space, and let $P\left( X \right)$ denote the
  collection of non-empty closed subsets of $X$. For $A$ and $B$ in $P\left( X \right)$, let
  \begin{equation}
    \notag
    h\left( A, B \right) = \sup_{x\in X} \left| \dist(x, A) - \dist(x, B) \right|,
  \end{equation}
  where $\dist(x, C)$ is given by
  \begin{equation}
    \notag
    \dist\left( x, C \right) = \inf_{c\in C}d(x, c).
  \end{equation}
  The function $h$ is called the \textit{Hausdorff metric}.
  \begin{enumerate}
    \item Show that if $h(A, B) = 0$ then $A = B$. Here $A$ and $B$ are two non-empty closed subsets of $X$.
    \item Show that $h$ is a metric on $P\left( X \right)$.
    \item For $A$ and $B$ in $P\left( X \right)$, let $\hat{h}$ be defined as
      \begin{equation}
        \notag
        \hat{h}\left( A, B \right) = \max \left\{ \sup_{a \in A}\dist\left( a, B \right), \sup_{b\in B}\dist\left( b, A\right) \right\}.
      \end{equation}
  \end{enumerate}
  Show that
  \begin{equation}
    \notag
    \hat{h}\left( A,B \right) = h\left( A, B \right) \text{ for all } A, B \text{ in } P\left( X \right).
  \end{equation}
  (\textbf{Hint:} Show the two inequalities $h\left( A, B \right) \geq \hat{h}\left( A, B \right)$ and $\hat{h}\left( A, B \right) = h\left( A, B \right)$.) 
\end{prb}

\begin{sltn}
  Before I start, I want to jot down the properties of the various mathematical
  objects presented.  We are given a metric space $\left( X, d \right)$ and it
  is said to be bounded. What this means, is that there exists some number $r$
  such that $d(x, y) \leq r$ for all $x, y \in X$.  $P\left( X \right)$ is a
  collection of non-empty closed subsets of $X$.  In other words, the elements
  of $P\left( X \right)$ are sets that contain their own boundary.

  We want to show that $h\left( A, B \right) = 0 \rimply A = B$. Let us therefore assume that
  $h\left( A, B \right) = 0$. This gives us the following equation:
  \begin{equation}
    \notag
    \sup_{x\in X}\left|\dist(x, A) - \dist(x, B)\right| = 0. 
  \end{equation}
  This means that the largest difference we can have between $\dist(x, A)$ and
  $\dist(x, B)$ is $0$, however, since we are working with absolute values this
  actually means that the $\dist(x, A) - \dist(x, B) = 0$ for all $x \in X$.
  We can therefore meaningfully examine the equation, 
  \begin{equation}
    \notag
    \dist(x, A) = \dist(x, B).
  \end{equation}
  This equation being true means that if you pick the element in $A$ that is the smallest
  distance away from $x$ then this distance is exactly equal to the distance
  between the element in $B$ that is the smallest distance away from $x$. Or, strictly
  speaking
  \begin{equation}
    \notag
    \inf_{a \in A} d(x, a) = \inf_{b \in B} d(x, b). 
  \end{equation}


\end{sltn}

\begin{prb}
  Let $0 < r < 1$ and consider the series
  \begin{equation}
    \notag
    \sum_{n=-\infty}^{\infty}r^{|n|}e^{inx}.
  \end{equation}
  Show that the series converges uniformly for all $x \in \mathbb{R}$, and that
  its sum equals
  \begin{equation}
    \notag
    P_{r}(x) = \frac{1-r^{2}}{1 - 2r \cos(x)+r^{2}}.
  \end{equation}
\end{prb}
\end{document}

