%        File: proofs.tex
%     Created: Mon Mar 16 10:00 AM 2015 C
% Last Change: Mon Mar 16 10:00 AM 2015 C
%

\documentclass[a4paper]{report}


\usepackage[utf8]{inputenc} 

\usepackage[]{amsmath} 
\usepackage[]{amsthm} 
\usepackage[]{amssymb} 
\usepackage[]{enumerate} 

\newtheorem{prp}{Proposition}
\newtheorem{thrm}{Theorem}
\newtheorem{lma}{Lemma}
\newtheorem{crl}{Corollary}
\theoremstyle{definition}
\newtheorem{defn}{Definition}

\begin{document}

\chapter{Metric Spaces}
\section{Definitions and examples}

\begin{prp}[Inverse Triangle Inequality]
  For all elements $x, y, z$ in a metric space $\left(X, d\right)$, we have:
  \begin{equation}
    \notag
    |d(x, y) - d(x, z)| \leq |d(y, z)|
  \end{equation}
\end{prp}

\begin{proof}
  Let $(X, d)$ be a metric space, and let $x, y, z$ be three aribtrary elements
  in $X$.  From the triangle inequality, that we know $d$ satisfy from the
  definition of a metric space, we have:
  \begin{equation}
    \notag
    d(x, y) \leq d(x, z) + d(z, y).
  \end{equation}

  Now, in order to show that the inverse triangle inequality holds, note that
  the absolute value involved is going to be the largest of the two numbers
  $d(x, y) - d(x, z)$ and $d(x, z) - d(x, y)$. It therefore suffices to show
  that both of these must be smaller than $d(y, z)$. 
  
  Simply rearranging the triangle inequality shows us that the first inequality
  holds.
  \begin{equation}
    \notag
    d(x, y) - d(x, z) \leq d(y, z)
  \end{equation}
  Applying the triangle inequality to the points $x, z$ with $y$ as an
  intermediate point gives us
  \begin{equation}
    \notag
    d(x, z) \leq d(x, y) + d(y, z),
  \end{equation}
  and this can be rearranged to give the second inequality.
\end{proof}

\section{Convergence and continuity}
\begin{defn}
  Let $(X, d)$ be a metric space. A sequence $\left\{ x_n \right\}$ in $X$
  converges to a point $a \in X$ if there for every $\varepsilon > 0$ exists an
  $N \in \mathbb{N}$ such that $d(x_{n}, a) < \varepsilon$ for all $n \geq N$.
  We write $\lim\limits_{n\rightarrow \infty} x_n = a$ or $x_{n} \rightarrow a$.
\end{defn}

\begin{lma}
  A sequence $\left\{ x_{n}\right\}$ in a metric space $(X, d)$ converges to
  $a$ if and only if $\lim\limits_{n\rightarrow\infty} d(x_{n}, a) = 0$.
\end{lma}
\begin{proof}
  This is simply a reformulation of the previous definition, but we prove it
  rigorously by showing that these statements are equivalent.
  Let $(X, d)$ be a metric space, and let $\left\{ x_n \right\}$ be a sequence
  in this metric space.  
  
  Let $a \in X$ and assume that $\left\{ x_n \right\}$ converges to $a$. By
  definition of convergence, for every $\varepsilon > 0$ we can chose an $N \in
  \mathbb{N}$ such that $d(x_{n}, a) < \varepsilon$ for all $n \geq N$.  This
  simply means that we can force the distance between the elements in $\left\{
  x_{n} \right\}$ and $a$ to be arbitrarily close to zero by picking elements
  far out in the sequence. 
  
  We can generate a new sequence of the distances
  between the elements of $\left\{ x_n \right\}$ and $a$, namely the sequence
  $\left\{d(x_n, a)\right\}$.  Since we have $\lim\limits_{n\rightarrow\infty} x_n
  = a$ we know that the following limit must equate to zero:
  \begin{equation}
    \notag
    \lim\limits_{n\rightarrow\infty}d(x_n, a) = d(a, a) = 0.
  \end{equation}

  Now, assume that $\lim\limits_{n\rightarrow\infty} d(x_{n}, a) = 0$.  For
  this equation to be true, we must have $\lim\limits_{n\rightarrow\infty}x_n =
  a$, because we have equality only when $x_n = a$, by the definition of a
  metric.  But then, by definition, the sequence $\left\{ x_n \right\}$
  converges to $a$.
\end{proof}

\begin{prp}
  A sequence in a metric space can not converge to more than one point. 
\end{prp}

\begin{proof}
  Let $\left\{ x_n \right\}$ be a sequence in a metric space $(X, d)$. Assume
  for contradiction that $\left\{ x_{n} \right\}$ converges to both the point
  $a$ and the point $a'$.  By definition of convergence, we have
  $\lim\limits_{n\rightarrow\infty} x_n = a$ and
  $\lim\limits_{n\rightarrow\infty} x_n = a'$.
  Using the triangle inequality we have:
  \begin{equation}
    \notag
    d(a, a') \leq d(a, x_n) + d(x_n, a').
  \end{equation}
  Taking the limits we get the following inequality:
  \begin{equation}
    \notag
    d(a, a') \leq \lim\limits_{n\rightarrow\infty} d(a, x_n) + \lim\limits_{n\rightarrow\infty} d(x_n, a') = 0 + 0 = 0, 
  \end{equation}
  but this is only possible if $a = a'$. (The limits equal zero, due to the previous lemma.)
\end{proof}

\begin{defn}
  Assume that $(X, d_{X})$ and $(Y, d_{Y})$ are two metric spaces. A function
  $f : X \rightarrow Y$ is \textit{continuous} at a point $a \in X$ if for
  every $\varepsilon > 0$ there is a $\delta > 0$ such that $d_{Y}(f(x), f(a))
  < \varepsilon$ whenever $d_{X}(x, a) < \delta$.
\end{defn}

\begin{prp}
  The following are equivalent for a function $f : X \rightarrow Y$ between metric spaces:
  \begin{enumerate}[(i)]
    \item $f$ is continuous at a point $a \in X$.
    \item For all sequences $\left\{ x_n \right\}$ converging to $a$, the sequence $\left\{ f(x_n) \right\}$ converges
      to $f(a)$.
  \end{enumerate}
\end{prp}

\begin{proof}
  We show that this is true by showing both the left and right implication.
  Let us first assume that $f$ is continuous at a point $a \in X$. By
  definition of continuity we have that for all $\varepsilon > 0$ there is a
  $\delta > 0$ such that $d(f(x), f(a)) < \varepsilon$ whenever $d(x, a) <
  \delta$.

  Assume now that an arbitrary sequence $\left\{ x_n \right\}$ converges to
  $a$. This means that $\lim\limits_{n\rightarrow\infty} x_n = a$.  Therefore,
  we must have $\lim\limits_{n\rightarrow\infty} f(x_n) = f(a)$ since functions
  obey the axiom of substitiom from zet theory. Thus the right implication is shown.
  We now need to show the left implication.

  Assume that for all sequences $\left\{ x_n \right\}$ converging to $a$, the
  sequence $\left\{ f(x_n) \right\}$ converges to $f(a)$.  We must now show
  that this implies that $f$ is continuous at $a$. Since we have $f(x_n) \rightarrow f(a)$
  there exists, for all $\varepsilon > 0$ an $N \in \mathbb{N}$ such that picking elements farther
  than $N$ into the sequence, we can get the distance between these elements and $f(a)$ arbitrarily small.
  But this is the definition of $f$ being continuous at the point $a$.
\end{proof}
\end{document}


