%        File: proofs.tex
%     Created: Mon Mar 16 10:00 AM 2015 C
% Last Change: Mon Mar 16 10:00 AM 2015 C
%

\documentclass[a4paper]{report}


\usepackage[utf8]{inputenc} 

\usepackage[]{amsmath} 
\usepackage[]{amsthm} 
\usepackage[]{amssymb} 
\usepackage[]{enumerate} 

\newtheorem{prp}{Proposition}
\newtheorem{thrm}{Theorem}
\newtheorem{lma}{Lemma}
\newtheorem{crl}{Corollary}
\theoremstyle{definition}
\newtheorem{defn}{Definition}

\begin{document}
\chapter{Metric Spaces}
\section{Definitions and examples}

\begin{prp}[Inverse Triangle Inequality]
  For all elements $x, y, z$ in a metric space $\left(X, d\right)$, we have:
  \begin{equation}
    \notag
    |d(x, y) - d(x, z)| \leq |d(y, z)|
  \end{equation}
\end{prp}

\begin{proof}
  Let $(X, d)$ be a metric space, and let $x, y, z$ be three aribtrary elements
  in $X$.  From the triangle inequality, that we know $d$ satisfy from the
  definition of a metric space, we have:
  \begin{equation}
    \notag
    d(x, y) \leq d(x, z) + d(z, y).
  \end{equation}

  Now, in order to show that the inverse triangle inequality holds, note that
  the absolute value involved is going to be the largest of the two numbers
  $d(x, y) - d(x, z)$ and $d(x, z) - d(x, y)$. It therefore suffices to show
  that both of these must be smaller than $d(y, z)$. 

  Simply rearranging the triangle inequality shows us that the first inequality
  holds.
  \begin{equation}
    \notag
    d(x, y) - d(x, z) \leq d(y, z)
  \end{equation}
  Applying the triangle inequality to the points $x, z$ with $y$ as an
  intermediate point gives us
  \begin{equation}
    \notag
    d(x, z) \leq d(x, y) + d(y, z),
  \end{equation}
  and this can be rearranged to give the second inequality.
\end{proof}

\section{Convergence and continuity}
\begin{defn}
  Let $(X, d)$ be a metric space. A sequence $\left\{ x_n \right\}$ in $X$
  converges to a point $a \in X$ if there for every $\varepsilon > 0$ exists an
  $N \in \mathbb{N}$ such that $d(x_{n}, a) < \varepsilon$ for all $n \geq N$.
  We write $\lim\limits_{n\rightarrow \infty} x_n = a$ or $x_{n} \rightarrow a$.
\end{defn}

\begin{lma}
  A sequence $\left\{ x_{n}\right\}$ in a metric space $(X, d)$ converges to
  $a$ if and only if $\lim\limits_{n\rightarrow\infty} d(x_{n}, a) = 0$.
\end{lma}
\begin{proof}
  This is simply a reformulation of the previous definition, but we prove it
  rigorously by showing that these statements are equivalent.
  Let $(X, d)$ be a metric space, and let $\left\{ x_n \right\}$ be a sequence
  in this metric space.  

  Let $a \in X$ and assume that $\left\{ x_n \right\}$ converges to $a$. By
  definition of convergence, for every $\varepsilon > 0$ we can chose an $N \in
  \mathbb{N}$ such that $d(x_{n}, a) < \varepsilon$ for all $n \geq N$.  This
  simply means that we can force the distance between the elements in $\left\{
  x_{n} \right\}$ and $a$ to be arbitrarily close to zero by picking elements
  far out in the sequence. 

  We can generate a new sequence of the distances
  between the elements of $\left\{ x_n \right\}$ and $a$, namely the sequence
  $\left\{d(x_n, a)\right\}$.  Since we have $\lim\limits_{n\rightarrow\infty} x_n
  = a$ we know that the following limit must equate to zero:
  \begin{equation}
    \notag
    \lim\limits_{n\rightarrow\infty}d(x_n, a) = d(a, a) = 0.
  \end{equation}

  Now, assume that $\lim\limits_{n\rightarrow\infty} d(x_{n}, a) = 0$.  For
  this equation to be true, we must have $\lim\limits_{n\rightarrow\infty}x_n =
  a$, because we have equality only when $x_n = a$, by the definition of a
  metric.  But then, by definition, the sequence $\left\{ x_n \right\}$
  converges to $a$.
\end{proof}

\begin{prp}
  A sequence in a metric space can not converge to more than one point. 
\end{prp}

\begin{proof}
  Let $\left\{ x_n \right\}$ be a sequence in a metric space $(X, d)$. Assume
  for contradiction that $\left\{ x_{n} \right\}$ converges to both the point
  $a$ and the point $a'$.  By definition of convergence, we have
  $\lim\limits_{n\rightarrow\infty} x_n = a$ and
  $\lim\limits_{n\rightarrow\infty} x_n = a'$.
  Using the triangle inequality we have:
  \begin{equation}
    \notag
    d(a, a') \leq d(a, x_n) + d(x_n, a').
  \end{equation}
  Taking the limits we get the following inequality:
  \begin{equation}
    \notag
    d(a, a') \leq \lim\limits_{n\rightarrow\infty} d(a, x_n) + \lim\limits_{n\rightarrow\infty} d(x_n, a') = 0 + 0 = 0, 
  \end{equation}
  but this is only possible if $a = a'$. (The limits equal zero, due to the previous lemma.)
\end{proof}

\begin{defn}
  Assume that $(X, d_{X})$ and $(Y, d_{Y})$ are two metric spaces. A function
  $f : X \rightarrow Y$ is \textit{continuous} at a point $a \in X$ if for
  every $\varepsilon > 0$ there is a $\delta > 0$ such that $d_{Y}(f(x), f(a))
  < \varepsilon$ whenever $d_{X}(x, a) < \delta$.
\end{defn}

\begin{prp}
  The following are equivalent for a function $f : X \rightarrow Y$ between metric spaces:
  \begin{enumerate}[(i)]
    \item $f$ is continuous at a point $a \in X$.
    \item For all sequences $\left\{ x_n \right\}$ converging to $a$, the sequence $\left\{ f(x_n) \right\}$ converges
      to $f(a)$.
  \end{enumerate}
\end{prp}

\begin{proof}
  We show that this is true by showing both the left and right implication.
  Let us first assume that $f$ is continuous at a point $a \in X$. By
  definition of continuity we have that for all $\varepsilon > 0$ there is a
  $\delta > 0$ such that $d(f(x), f(a)) < \varepsilon$ whenever $d(x, a) <
  \delta$.

  Assume now that an arbitrary sequence $\left\{ x_n \right\}$ converges to
  $a$. This means that $\lim\limits_{n\rightarrow\infty} x_n = a$.  Therefore,
  we must have $\lim\limits_{n\rightarrow\infty} f(x_n) = f(a)$ since functions
  obey the axiom of substitiom from zet theory. Thus the right implication is shown.
  We now need to show the left implication.

  Assume that for all sequences $\left\{ x_n \right\}$ converging to $a$, the
  sequence $\left\{ f(x_n) \right\}$ converges to $f(a)$.  We must now show
  that this implies that $f$ is continuous at $a$. Since we have $f(x_n) \rightarrow f(a)$
  there exists, for all $\varepsilon > 0$ an $N \in \mathbb{N}$ such that picking elements farther
  than $N$ into the sequence, we can get the distance between these elements and $f(a)$ arbitrarily small.
  But this is the definition of $f$ being continuous at the point $a$.
\end{proof}

\chapter{Preliminaries}
\chapter{Spaces of continuous functions}

\section{The spaces $C(X, Y)$}


\begin{lma}
  \label{lma_331} 
  Let $(X, d_X)$ and $(Y, d_Y)$ be metric spaces, and assume that $X$ is
  compact. If $f, g : X \rightarrow Y$ are continuous functions, then
  \begin{equation}
    \notag
    \rho(f, g) = \sup\left\{ d_{Y}\left( f(x), g(x) \right) \mid x \in X
      \right\}\footnote{The basic idea is to measure the distance between two
    functions by looking at the point they are the furthest apart}.
  \end{equation}
  is finite, and there is a point $x \in X$ such that $d_{Y}\left( f(x), g(x)
  \right) = \rho(f, g)$.
\end{lma}
\begin{proof}

  The condition that $X$ is compact is crucial in this situation, because it
  allows us to make use of the extreme value theorem. This theorem states that
  if $X$ is compact, and if the function with $X$ as its domain is continuous,
  then said function has a maximum and minimum point in $X$. 

  We introduce the function
  \begin{equation}
    \notag
    h(x) = d_Y(f(x), g(x)).
  \end{equation}
  We need to show that this function is continuous. We do this by showing that
  $|h(x) - h(y)|$ can be forced to be less than $\varepsilon$.  Using the
  triangle inequality and the inverse triangle inequality we get
  \begin{align*}
    |h(x) - h(y)| &= | d_{Y}\left( f(x), g(x) \right) - d_{Y}\left( f(y), g(y) \right) | \\
    &= | d_{Y}\left( f(x), g(x) \right) - d_{Y}\left( f(x), g(y) \right) + d_{Y}\left( f(x), g(y) \right) - d_{Y}\left( f(y), g(y) \right) | \\
    &\leq | d_{Y}\left( f(x), g(x) \right) - d_{Y}\left( f(x), g(y) \right)| + |d_{Y}\left( f(x), g(y) \right) - d_{Y}\left( f(y), g(y) \right) | \\
    &\leq d_{Y}\left( g(x), g(y) \right) + d_{Y}\left( f(x), f(y) \right)
  \end{align*}
  Since both $f$ and $g$ are continuous functions, we can always chose a
  $\delta > 0$ such that $\leq d_{Y}\left( g(x), g(y) \right) < \varepsilon/2$
  and $ d_{Y}\left( f(x), f(y) \right) < \varepsilon/2$.

  We therefore have
  \begin{equation}
    \notag
    |h(x) - h(y)| \leq d_{Y}\left( g(x), g(y) \right) + d_{Y}\left( f(x), f(y)
    \right) < \varepsilon/2 + \varepsilon/2 = \varepsilon, 
  \end{equation}
  and by the extreme value theorem $p(f, g)$ is finite and there is a point $x
  \in X$ such that $d_{Y}\left( f(x), g(x) \right) = \rho(f, g)$.

\end{proof}
We can now show that $\rho$ is a metric on $C(X, Y)$.

\begin{prp}
  \label{prp_332}
  Let $(X, d_X)$ and $(Y, d_Y)$ be metric spaces, and assume that $X$ is
  compact. Then
  \begin{equation}
    \notag
    \rho(f, g) = \sup\left\{ d_{Y}\left( f(x), g(x) \right) \mid x \in X\right\}
  \end{equation}
  defines a metric on $C\left( X, Y \right)$.
\end{prp}
\begin{proof}
  We need to show symmetry, positivity and the triangle inequality.  
  \begin{enumerate}
    \item Symmetry \\
      [0.2cm]
      We need to show that $\rho(f, g) = \rho(g, f)$. Since $d_Y$ is a metric,
      we have $d_Y(f(x), g(x)) = d_Y(g(x), f(x))$ therefore this property
      follows from the $d_Y$ metric.

    \item Positivity \\
      [0.2cm]
      Again, this follows from the metric $d_Y$.     

    \item Triangle inequality \\
      [0.2cm]
      Need to show that $\rho(f, g) \leq \rho(f, h) + \rho(h, g)$.
      By Lemma \ref{lma_331}, there is an $x \in X$ such that $\rho(f, g) = d_Y(f(x), g(x))$. 
      We therefore have
      \begin{align*}
        \rho(f, g) &= d_Y(f(x), g(x)) \\
        &\leq d_Y(f(x), h(x)) + d_Y(h(x), g(x)) \\
        &= \rho(f, h) + \rho(h, g).
      \end{align*}
      as we wanted to show.
  \end{enumerate}
\end{proof}

\begin{prp}
  \label{prp_333}
  A sequence $\left\{ f_n \right\}$ converges to $f$ in $\left( C(X, Y), \rho
  \right)$ if and only if it converges uniformly to $f$.
\end{prp}
\begin{proof}
  By Proposition 3.2.3, $\left\{ f_n \right\}$ converges uniformly to $f$ if
  and only if 
  \begin{equation}
    \notag
    \sup\left\{ d_Y(f_n(x), f(x)) \right\} \rightarrow 0, 
  \end{equation}
  as $n \rightarrow \infty$. But this is just saying that $\rho(f_n, f) \rightarrow
  0$ as $n \rightarrow \infty$.
\end{proof}

\begin{thrm}
  \label{thm_334}
  Assume that $\left( X, d_X \right)$ is a compact and $(Y, d_Y)$ a complete
  metric space. Then $\left( C(X, Y), \rho \right)$ is complete. 
\end{thrm}
\begin{proof}
  Assume that $\left\{ f_n \right\}$ is a Cauchy sequence in $C(X, Y)$. We must
  prove that it converges, by the definition of a complete space, to a function
  $f \in C(X, Y)$.
\end{proof}

\section{Applications to differential equations}
\section{Compact subsets of $C(X, \mathbb{R}^{m})$}
\begin{defn}
  \label{defn_351}
  Let $(X, d)$ be a metric space and assume that $A$ is a subset of $X$. We say
  that $A$ is \textit{dense} in $X$ if for each $x \in X$ there is a sequence from $A$
  converging to $x$.
\end{defn}
\begin{defn}
  \label{defn_352} 
  A metric set $(X, d)$ is called \textit{separable} if it has a countable,
  dense subset $A$.
\end{defn}

\begin{prp}
  \label{prp_353} 
  All compact metric spaces $(X, d)$ are separable. We can choose the countable
  dense set $A$ in such a way that for any $\delta > 0$, there is a finite
  subset $A_{\delta}$ of $A$ such that all elements of $X$ are within distance
  less than $\delta$ of $A_{\delta}$, i.e., for all $x \in X$ there is an $a
  \in A_{\delta}$ such that $d(x, a) < \delta$. 
\end{prp}

\begin{proof}
  We need to show that $(X, d)$ has a countable dense subset $A$.  We know that
  by definition, a compact set $X$ is totally bounded.  This means that there
  exists a finite number of subsets of $X$ whose union contains $X$. More
  formally, for all $n \in \mathbb{N}$ there is a finite number of balls of
  radius $\frac{1}{n}$ that cover $X$. The centers of these balls constitute a
  countable subset of $X$, lets call it $A$. 

  All we need to do now, is show that this countable subset $A$ of $X$ is also
  dense.  For $A$ to be dense, we need that for each $x \in X$ there is a
  sequence from $A$ converging to $x$. Let $x$ be an element of $X$. We need to
  find a sequence $\left\{ a_n \right\}$ that converges to $x$ in $A$. 
  
  That is $d(\left\{ a_n \right\}, x) \rightarrow 0$ as $n \rightarrow \infty$.
  First we pick the center $a_1$ of one of the balls of radius $1$ that $x$
  belongs to. Then we pick the center $a_2$ of one of the balls of radius
  $\frac{1}{2}$ that $x$ belongs to, and so on and so forth. To find the set
  $A_\delta$ just chose $m \in \mathbb{N}$ so big that $\frac{1}{m} < \delta$
  and let $A_\delta$ consist of the centers of the balls of radius
  $\frac{1}{m}$.
\end{proof}
\end{document}


